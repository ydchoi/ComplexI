\documentclass{article} % For LaTeX2e
\usepackage{nips14submit_e,times}
\usepackage{amsmath}
\usepackage{amsthm}
\usepackage{amssymb}
\usepackage{mathtools}
\usepackage{hyperref}
\usepackage{url}
\usepackage{algorithm}
\usepackage[noend]{algpseudocode}
%\documentstyle[nips14submit_09,times,art10]{article} % For LaTeX 2.09

\usepackage{graphicx}
\usepackage{caption}
\usepackage{subcaption}

\def\eQb#1\eQe{\begin{eqnarray*}#1\end{eqnarray*}}
\def\eQnb#1\eQne{\begin{eqnarray}#1\end{eqnarray}}
\providecommand{\e}[1]{\ensuremath{\times 10^{#1}}}
\providecommand{\pb}[0]{\pagebreak}

\newcommand{\E}{\mathrm{E}}
\newcommand{\Var}{\mathrm{Var}}
\newcommand{\Cov}{\mathrm{Cov}}

\def\Qb#1\Qe{\begin{question}#1\end{question}}
\def\Sb#1\Se{\begin{solution}#1\end{solution}}

\newenvironment{claim}[1]{\par\noindent\underline{Claim:}\space#1}{}
\newtheoremstyle{quest}{\topsep}{\topsep}{}{}{\bfseries}{}{ }{\thmname{#1}\thmnote{ #3}.}
\theoremstyle{quest}
\newtheorem*{definition}{Definition}
\newtheorem*{theorem}{Theorem}
\newtheorem*{lemma}{Lemma}
\newtheorem*{question}{Question}
\newtheorem*{preposition}{Preposition}
\newtheorem*{exercise}{Exercise}
\newtheorem*{challengeproblem}{Challenge Problem}
\newtheorem*{solution}{Solution}
\newtheorem*{remark}{Remark}
\usepackage{verbatimbox}
\usepackage{listings}
\title{Complex Analysis I: \\
Problem Set I}


\author{
Youngduck Choi \\
CILVR Lab \\
New York University\\
\texttt{yc1104@nyu.edu} \\
}


% The \author macro works with any number of authors. There are two commands
% used to separate the names and addresses of multiple authors: \And and \AND.
%
% Using \And between authors leaves it to \LaTeX{} to determine where to break
% the lines. Using \AND forces a linebreak at that point. So, if \LaTeX{}
% puts 3 of 4 authors names on the first line, and the last on the second
% line, try using \AND instead of \And before the third author name.

\newcommand{\fix}{\marginpar{FIX}}
\newcommand{\new}{\marginpar{NEW}}

\nipsfinalcopy % Uncomment for camera-ready version

\begin{document}


\maketitle

\begin{abstract}
This work contains the solutions to the problem set I
of Complex Analysis I 2015 at Courant Institute of Mathematical Sciences.
\end{abstract}

\begin{question}[1. Brown p.79-1]
\end{question}
\begin{solution}
The Cauchy-Riemann equation in polar coordinates can be given by
\eQb
u_{\theta} = -rv_r \text{ and } ru_r = v_{\theta}. 
\eQe
Differentiating the first equation with respect to the $\theta$ parameter, we obtain
\eQnb
u_{\theta \theta} &=& - rv_{r\theta}. 
\eQne
Differentiating the second equation with respect to the $r$ parameter, we have
\eQnb
v_{\theta r} &=& u_{r} + ru_{rr}.
\eQne
Multiplying $(2)$ by $r$ and subtracting with $(1)$, we obtain
\eQb
-u_{\theta \theta} + rv_{\theta r} = rv_{r\theta} +  ru_{rr} + r^2u_{rr}.
\eQe
As $v_{r \theta} = v_{\theta r}$ by Fubini's theorem, we obtain
\eQb
r^2u_{rr} + ru_r + u_{\theta \theta} &=& 0.
\eQe
Therefore, we have derived the polar form
of Laplace's equation. Now, we wish to verify that the same equation holds for $v$.
Consider the expression $r^2v_{rr} + rv_{r} + v_{\theta \theta}$. Differentiating 
$v_r = -\dfrac{1}{r}u_{\theta}$ with respect to $r$, we have 
\eQb
v_{rr} = \dfrac{1}{r^2}u_{\theta} - \dfrac{1}{r}u_{\theta r}.
\eQe
Differentiating $v_{\theta} = ru_r$ with respect to $\theta$, we have
\eQb
v_{\theta \theta} = ru_{r \theta}.
\eQe
Substituting the last two equations into the considered expression, we have
\eQb
u_{\theta} - ru_{\theta r} + u_{\theta} + ru_{r\theta},
\eQe
which again by Fubini equals $0$. $\qed$
\end{solution}

\pagebreak

\begin{question}[1-2. Brown p.79-2]
\end{question}
\begin{solution}
Let the function $f(z) = u(x,y) + iv(x,y)$ be analytic on a domain $D$,
and consider a point $z_0 = (x_0, y_0) = (c_1, c_2)$. Assume that $f'(z_0) \neq 0$.
For the level curve of $u(x,y) = c_1$, by implicit differentiation
with respect to $x$, we obtain 
\eQb
u_x + u_y \dfrac{dy}{dx} = 0. 
\eQe
As $f'(z_0) \neq 0$, the above equation can be simplified to 
\eQb
\dfrac{dy}{dx} = -\dfrac{u_x}{u_y}.
\eQe
Analogously, for the corresponding 
level curve of $v(x,y) = c_2$, by implicit differentiation
with respect to $x$, we obtain
\eQb
v_x + v_y \dfrac{dy}{dx} &=& 0,
\eQe
As $f'(z_0) \neq 0$, the above equation can be simplified to 
\eQb
\dfrac{dy}{dx} &=& -\dfrac{v_x}{v_y}.
\eQe
Now, as the function is analytic in a domain $D$, the Cauchy-Riemann equation must hold 
at $z_0$. Hence, the last equation can be extened as
\eQb
\dfrac{dy}{dx} &=& -\dfrac{v_x}{v_y} = \dfrac{u_y}{u_x}. 
\eQe
Therefore, we have shown that the level curves of the component functions
at the analytic domain $D$ are orthogonal. $\qed$
\end{solution}

\bigskip


\begin{question}[1-3. Brown p.79-4]
\end{question}
\begin{solution}

\end{solution}

\bigskip

\begin{question}[2. Conjugate Harmonic Function]
\end{question}
\begin{solution}
We are given a polynomial of the following form:
\eQb
u(x,y) &=& ax^3 + bx^{2}y + cxy^{2} + dy^{3}.
\eQe
Differentiating the given polynomial $u(x,y)$ with respect to $x$ and $y$ variables, we obtain
\eQb
\dfrac{\partial u}{\partial x} = 3ax^2 + 2byx + cy^{2}, \\ 
\dfrac{\partial^2 u}{\partial x^2} = 6ax + 2by, \\ 
\dfrac{\partial u}{\partial y} = 3dy^2 + 2cxy + bx^{2}, \\ 
\dfrac{\partial^2 u}{\partial y^2} = 6dy + 2cx. \\ 
\eQe
As the given polynomial is harmonic, $\dfrac{\partial^2 u}{\partial x^2} + 
\dfrac{\partial^2 u}{\partial y^2} = 0$ must hold for all $x$ and $y$.
Hence, we obtain
\eQb
6ax + 2by + 6dy + 2cx = 0,
\eQe
which simplifies to 
\eQb
(3a + c)x + (3d + b)y = 0,
\eQe
for all $x$ and $y$. As the above equation
must hold for all $x$ and $y$, we have that
\eQb
c = -3a \text{ and } b = -3d.
\eQe
Substituting the above equations into the original polynomial, we obtain the 
most general harmonic polynomial of the given form as 
\eQb
u(x,y) = ax^3 - 3dx^2y - 3axy^2 + dy^3.
\eQe
Now, we wish to compute the conjugate harmonic function of $u(x,y)$, denoted by $v(x,y)$.
From the Cauchy-Riemann equations, we have
\eQb
\dfrac{\partial u}{\partial x} = \dfrac{\partial v}{\partial y} = 3ax^2 + 2byx + cy^2, \\
\dfrac{\partial u}{\partial y} = -\dfrac{\partial v}{\partial x} = 3dy^2 + 2cxy + bx^2.
\eQe
Simplifying for the $v(x,y)$ term, we have
\eQb
\dfrac{\partial v}{\partial y} = 3ax^2 + 2byx + cy^2, \\
\dfrac{\partial v}{\partial x} = -(3dy^2 + 2cxy + bx^2).
\eQe
By taking the integral, we can solve for $v(x,v)$, and obtain
\eQb
v(x,y) &=& \dfrac{1}{3}cy^3 - \dfrac{1}{3}bx^3 + bxy^2 - cx^2y 
\eQe


\end{solution}

\bigskip

\begin{question}[3. The Complex Chain Rule]
\end{question}
\begin{solution}
\end{solution}

\bigskip

\begin{question}[4. $|f(z)| = 1$ implies constant $f$]
\end{question}
\begin{solution}
Let $D$ be a unit disk, and $f:D \to \mathbb{C}$ be a holomorphic function such that
$|f(z)| = 1$ for $z \in D$. Notice that $|f|$ is also a holomorhpic function, which
can be re-written as $u(x,y) + iv(x,y)$. As $|f(z)| = 1$ for all $z \in D$, we obtain that
$u^2 + v^2 = 1$ for all $(x,y) \in D$.
Taking the partials, we obtain
\eQb
2uu_x + 2vv_x = 0 &\text{ and } &
2uu_y + 2vv_y = 0.
\eQe
As $f$ is holomorphic, from the Cauchy-Riemann equation, we have that
\eQb
u_x = v_y &\text{ and } & u_y = -v_x. 
\eQe
Substituting the above equations to the partial equations and simplifying, 
we obtain
\eQb
uv_y + vv_x = 0 &\text{ and} & -uv_x + vv_y = 0, \\
u^2{v_y}^2 + v^2{v_x}^2 + 2uvv_xv_y = 0 &\text{ and }& u^2{v_x}^2 + v^2{v_y}^2 - 2uvv_x v_y. \\
\eQe
Hence, by adding the last two equations together, and factoring, we obtain
\eQb
(u^2 + v^2)({v_x}^2 + {v_y}^2) = 0.
\eQe
Since $|f| = 1$, $(u^2 + v^2)$ term cannot be $0$, and we obtain that
$v_x = 0$ and $v_y = 0$. By Cauchy-Riemann equation, we also have $u_x = 0$ and $u_y = 0$.
Thus, using the Cauchy-Riemann theorem, $f'(z)$ for $z \in D$ can be written as
\eQb
f'(z) &=& u_x + iv_x \\ 
&=& 0.
\eQe
Hence, $f'(z) = 0$ for all $z \in D$, and $f$ is a constant function. $\qed$
\end{solution}

\bigskip

\begin{question}[5. Harmonic Functions]
\end{question}
\begin{solution}
Let $u$ and $v$ be real harmonic function on the unit disk, and $f$ be a holomorphic
function on the unit disk. As $u$ and $v$ are harmonic, we have
\eQb
u_{xx} + u_{yy} = 0 &\text{ and }&
v_{xx} + v_{yy} = 0, 
\eQe
for all points on the unit disk. \\

\smallskip

\textbf{(i)} Assume that $u^2$ is harmonic. By using the chain rule, we can write $u^2_{xx}$
and $u^2_{yy}$ in terms of $u_x$ and $u_y$ respectively as 
\eQb
u^2_{xx} = 2(u_x u_x + uu_{xx}) &\text{ and } & u^2_{yy} = 2(u_y u_y + uu_{yy}).
\eQe
As $u^2$ is harmonic, we obtain
\eQb
2(u_x u_x + uu_{xx}) + 2(u_y u_y + uu_{yy}) &=& 0, \\
\eQe
which can be simplified and re-arranged to 
\eQb
u(u_{xx} + u_{yy}) + (u_x u_x + u_y u_y) &=& 0. 
\eQe
As $u_{xx} + u_{yy} = 0$, we finally obtain
\eQb
u_x u_x + u_y u_y &=& 0.
\eQe
Hence, $u_{x} = 0$, $u_{y} = 0$, and $u$ is a constant function. $\qed$ 

\smallskip

\textbf{(ii)} 
              
\end{solution}

\end{document}
