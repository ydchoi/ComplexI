\documentclass{article} % For LaTeX2e
\usepackage{nips14submit_e,times}
\usepackage{amsmath}
\usepackage{amsthm}
\usepackage{amssymb}
\usepackage{mathtools}
\usepackage{hyperref}
\usepackage{url}
\usepackage{algorithm}
\usepackage[noend]{algpseudocode}
%\documentstyle[nips14submit_09,times,art10]{article} % For LaTeX 2.09

\usepackage{graphicx}
\usepackage{caption}
\usepackage{subcaption}

\def\eQb#1\eQe{\begin{eqnarray*}#1\end{eqnarray*}}
\def\eQnb#1\eQne{\begin{eqnarray}#1\end{eqnarray}}
\providecommand{\e}[1]{\ensuremath{\times 10^{#1}}}
\providecommand{\pb}[0]{\pagebreak}

\newcommand{\E}{\mathrm{E}}
\newcommand{\Var}{\mathrm{Var}}
\newcommand{\Cov}{\mathrm{Cov}}

\def\Qb#1\Qe{\begin{question}#1\end{question}}
\def\Sb#1\Se{\begin{solution}#1\end{solution}}

\newenvironment{claim}[1]{\par\noindent\underline{Claim:}\space#1}{}
\newtheoremstyle{quest}{\topsep}{\topsep}{}{}{\bfseries}{}{ }{\thmname{#1}\thmnote{ #3}.}
\theoremstyle{quest}
\newtheorem*{definition}{Definition}
\newtheorem*{theorem}{Theorem}
\newtheorem*{lemma}{Lemma}
\newtheorem*{question}{Question}
\newtheorem*{preposition}{Preposition}
\newtheorem*{exercise}{Exercise}
\newtheorem*{challengeproblem}{Challenge Problem}
\newtheorem*{solution}{Solution}
\newtheorem*{remark}{Remark}
\usepackage{verbatimbox}
\usepackage{listings}
\title{Complex Analysis I: \\
Problem Set IV}


\author{
Youngduck Choi \\
CILVR Lab \\
New York University\\
\texttt{yc1104@nyu.edu} \\
}


% The \author macro works with any number of authors. There are two commands
% used to separate the names and addresses of multiple authors: \And and \AND.
%
% Using \And between authors leaves it to \LaTeX{} to determine where to break
% the lines. Using \AND forces a linebreak at that point. So, if \LaTeX{}
% puts 3 of 4 authors names on the first line, and the last on the second
% line, try using \AND instead of \And before the third author name.

\newcommand{\fix}{\marginpar{FIX}}
\newcommand{\new}{\marginpar{NEW}}

\nipsfinalcopy % Uncomment for camera-ready version

\begin{document}


\maketitle

\begin{abstract}
This work contains the solutions to the problem set IV
of Complex Analysis I 2015 at Courant Institute of Mathematical Sciences.
\end{abstract}

\bigskip

\begin{question}[1. Brown p.147-2]
\end{question}
\begin{solution}
\textbf{(b)} We first have that $\mathrm{cos}(z/2)$ is continous 
everywhere on the complex plane. Therefore, any contour from $0$
to $\pi + 2i$ will have the same value of $F(\pi + 2i) - F(0)$,
where $F$ denotes the antiderivative of $\mathrm{cos}(z/2)$.
We can compute the exact value as follows:
\eQb
\int_{0}^{\pi + 2i} \mathrm{cos}(\dfrac{z}{2}) dz
&=& \left[ 2\mathrm{sin}(\dfrac{z}{2}) \right]_0^{\pi + 2i}  \\
&=& 2\mathrm{sin}(\dfrac{\pi}{2} + i) \\
&=& 2\mathrm{cos}(i) \\
&=& e + \dfrac{1}{e},
\eQe
as desired. $\qed$
\end{solution}

\bigskip

\begin{question}[2. Brown p.147-5]
\end{question}
\begin{solution}
Let $C$ be a contour from $-1$ to $1$ that lies above the x-axis. We wish to compute the
following integral:
\eQb
\int_{C} z^i dz,
\eQe
where $z^i$ denotes the principal branch $exp(iLog(z))$ for $|z| > 0, -\pi < Arg(z) < 
\pi$. Notice that under the principal branch, $z = -1$ is not defined. The following
branch, however, agrees with the integrand along $C$ and is has anti-derivative along 
$C$:
\eQb
z^i = exp(ilog(z)) \text{ for } (|z| > 0, -\dfrac{\pi}{2} < arg(z) < \dfrac{3\pi}{2}).
\eQe
We then can compute the integral as follows:
\eQb
\int_{C} z^{i} dz &=& \left[ \dfrac{z^{i+1}}{i+1} \right]_{-1}^{1} \\
&=& \dfrac{1}{i+1}(e^{(i+1)log1} - e^{(i+1)log(-1)}) \\
&=& \dfrac{1}{i+1}(e^{(i+1)(ln1)} - e^{(i+1)ln(1 + i\pi)}) \\
&=& \dfrac{1}{i+1}(1 + e^{i\pi}) \\
&=& \dfrac{1 + e^{-\pi}}{2}(1 - i),
\eQe
as desired. $\qed$
\end{solution}

\bigskip

\begin{question}[3. Brown p.159-2]
\end{question}
\begin{solution}
\textbf{(b)}
Let $C_1$ denote the positively oriented boundary of the square 
whose sides lie along the line $x = \pm 1$, $y = \pm 1$ and 
let $C_2$ be the positively oriented circle $|z| = 4$. Observe that
$C_1$ is interior to $C_2$ and the given function 
$\dfrac{z+2}{\mathrm{sin}(\frac{z}{2})}$ is analytic in the closed
region consisting of the $C_1$ and $C_2$ contours and all points
between them. Hence, by the corollary, we have
\eQb
\int_{C_1} f(z) dz &=& \int_{C_2} f(z) dz,
\eQe
for $f(z) = \dfrac{z+2}{\mathrm{sin}(\frac{z}{2})}$. $\qed$
\end{solution}

\bigskip

\begin{question}[4. Brown 159-4]
\end{question}
\begin{solution}
\textbf{(a)}
Observe that along the lower horizontal leg, we have $z = x \>\> ( -a \leq x \leq a)$.
Hence, the integral along the lower horizontal leg from $-a$ to $a$ can be written as 
\eQb
2\int_{0}^{a} e^{-x^2} dx.
\eQe
For the upper horizontal leg, we have $z = x + ib \>\> (-a \leq x \leq a)$. Hence,
the integral along the upper horizontal from $a$ to $-a$ can be written as 
\eQb
\int_{a}^{-a} e^{-(x+ib)^2} dx,
\eQe
wich can be simplified as follows:
\eQb
\int_{a}^{-a} e^{-(x+ib)^2} dx &=& -e^{b^2}\int_{-a}^{a} e^{-x^2 - 2ibx} dx \\
&=& -2e^{b^2} \int_{0}^{a} e^{-x^2}\mathrm{cos}(2bx) dx. \\
\eQe
Hence, we have 
\eQb
2\int_{0}^{a} e^{-x^2} dx  
-2e^{b^2} \int_{0}^{a} e^{-x^2}\mathrm{cos}(2bx) dx,
\eQe
for the sum of contour integrals along each horizontal leg. 

\smallskip

Now, observe that 
along the vertical legs, we have $z = \pm a + iy \>\> ( 0 \leq y \leq b)$. Hence,
the integral along the vertical legs can be written as 
\eQb
I &=& \int_{0}^{b} e^{-(a+iy)^2} idy + \int_{b}^{0} e^{-(-a+iy)^2} idy, 
\eQe
which can be simplified as follows:
\eQb
I &=& \int_{0}^{b} e^{-a^2 - 2iay + y^2} idy - \int_{0}^{b} e^{-a^2 + 2iay + y^2} idy \\
&=& ie^{-a^2} \int_{0}^{b}e^{y^2}e^{-i2ay} - ie^{-a^2}\int_{0}^{b} e^{y^2}e^{2iay} dy. \\ 
\eQe
Hence, by the Cauchy-Goursat Theorem, we have 
\eQb
2\int_{0}^{a} e^{-x^2} dx  
-2e^{b^2} \int_{0}^{a} e^{-x^2}\mathrm{cos}(2bx) dx && \\
+ ie^{-a^2} \int_{0}^{b}e^{y^2}e^{-i2ay} - ie^{-a^2}\int_{0}^{b} e^{y^2}e^{2iay} dy &=& 0, \\
\eQe
which by re-arranging and simplifying becomes 
\eQb
\int_{0}^{a} e^{-x^2}cos(2bx)dx = e^{-b^2}\int_{0}^{a} e^{-x^2} dx + e^{-a^2-b^2}\int_{0}^{b}
e^{y^2}sin(2ay)dy,
\eQe
as desired.

\smallskip

\textbf{(b)} As $a \to \infty$, we have that
\eQb
e^{-a^2-b^2} \int_{0}^{b} e^{y^2} dy \to 0.
\eQe
Observe that 
\eQb
\left| e^{-a^2-b^2} \int_{0}^{b} e^{y^2}sin(2ay) dy \right|
\leq e^{-a^2-b^2} \int_{0}^{b} e^{y^2} dy.
\eQe
Hence, by the squeeze theorem, we have as $a \to \infty$
\eQb
e^{-a^2-b^2} \int_{0}^{b} e^{y^2}sin(2ay) dy \to 0.
\eQe
Therefore, with the given result in the problem, we obtain 
\eQb
\int_{0}^{\infty} e^{-x^2} cos(2bx) dx &=& \dfrac{\sqrt{\pi}}{2}e^{-b^2},
\eQe
as desired. $\qed$

\end{solution}

\bigskip

\begin{question}[5. Brown p.170-1]
\end{question}
\begin{solution}
Let $C$ denote the positively oriented boundary of the square whose sides
lie along the lines $x = \pm 2$ and $y = \pm 2$. We evaluate 
the following integrals. 

\smallskip

\textbf{(b)}
We are given the following integral:
\eQb
\int_{C} \dfrac{\mathrm{cos}(z)}{z(z^2+8)} dz,
\eQe
which can be written as
\eQb
\int_{C} \dfrac{\frac{\mathrm{cos}(z)}{(z^2+8)}}{z} dz.
\eQe
As $\dfrac{\mathrm{cos}(z)}{(z^2+8)}$ is analytic everywhere inside
and on the given contour, which is simple and closed, taken in the 
positive sense, by the Cauchy Integral formula, we obtain
\eQb
\dfrac{\mathrm{cos}(0)}{8} &=& \dfrac{1}{2\pi i} \int_{C} 
\dfrac{\frac{\mathrm{cos}(z)}{(z^2+8)}}{z} dz,
\eQe
which simplifies to 
\eQb
\int_{C} \dfrac{\frac{\mathrm{cos}(z)}{(z^2+8)}}{z} dz &=&
\dfrac{\pi i}{4}.
\eQe

\smallskip

\textbf{(d)}
We are given the following integral:
\eQb
\int_{C} \dfrac{\mathrm{cosh}(z)}{z^4} dz.
\eQe
As $\dfrac{\mathrm{cosh}(z)}{z^4}$ is analytic everywhere inside
and on the given contour, which is simple and closed, taken in the 
positive sense, by the extended Cauchy Integral formula, we obtain
\eQb
\mathrm{cosh}^{(3)}(z_0) 
&=& \dfrac{3!}{2\pi i} \int_{C} 
\dfrac{\mathrm{cosh}(z)}{(z-z_0)^4} dz,
\eQe
for $z_0$ inside and on the given contour. Observe that 
$\mathrm{cosh}^{(3)} = \mathrm{sinh}$. Hence, taking
$z_0 = 0$ yields
\eQb
0 &=& \dfrac{3!}{2\pi i} \int_{C} 
\dfrac{\mathrm{cosh}(z)}{z^4} dz,
\eQe  
which simplifies to 
\eQb
\int_{C} \dfrac{\mathrm{cosh}(z)}{z^4}dz &=& 0.
\eQe

\smallskip


\textbf{(e)} We are given the following integral:
\eQb
\int_{C} \dfrac{\mathrm{tan}(\frac{z}{2})}{(z-x_0)^2} dz, 
\eQe
for $-2 < x_0 < 2$. Notice that $x_0$ is inside the given contour. 
As $\dfrac{\mathrm{tan}(\frac{z}{2})}{(z-x_0)^2}$  is analytic
everywhere inside and on the given contour, which is simple and
closed, taken in the positive sense, by the extended Cauchy 
Integral formula, we obtain
\eQb
\dfrac{1}{2}\mathrm{sec}^{2}(\dfrac{x_0}{2}) &=& \dfrac{1!}{2\pi i} 
\int_{C} \dfrac{\mathrm{tan}(\frac{z}{2})}{(z-x_0)^2} dz, 
\eQe
which simplifies to 
\eQb
\int_{C} \dfrac{\mathrm{tan}(\frac{z}{2})}{(z-x_0)^2} dz 
&=& i \pi \mathrm{sec}^{2}(\frac{x_0}{2}),
\eQe
for $ -2 < x_0 < 2$. $\qed$
\end{solution}

\bigskip

\begin{question}[6. Brown 170.3]
\end{question}
\begin{solution}
Let $C$ be the circle $|z| = 3$, described in the positive sense.
As $2s^2 - s - 2$ is analytic everywhere inside and on the given contour,
which is simple and closed, taken in the positive sense, by the 
extended Cauchy Integral formula, we obtain 
\eQb
2z^2-z - 2 &=& \frac{1}{2\pi i} \int_{C} \dfrac{2s^2 - s - 2}{s-z} ds,
\eQe 
for $|z| < 3$. As $g(z) = \int_{C} \dfrac{2s^2 - s - 2}{s-z} ds$, we have
\eQb
g(z) = 2\pi i (2z^2 - z - 2),
\eQe 
for $|z| < 3$. Hence, it follows that $g(2) = 8\pi i$. For $|z| > 3$,
we have that $\dfrac{2s^2 - s -2 }{s - z}$ is analytic at all points 
interior to and on $C$. Hence, by the Cauchy-Goursat theorem, we obtain
\eQb
\int_{C} \dfrac{2s^2-s -2}{s-z} dz = 0,
\eQe 
for $|z| > 3$. Therefore, $g(z) = 0$ when $|z| > 3$,
which completes the solution for the problem. $\qed$
\end{solution}

\bigskip

\begin{question}[7. Brown 170-4]
\end{question}
\begin{solution}
Let $C$ be any simple closed contour, described in the positive sense
in the $z$ plane. As $s^3 + 2s$ is entire, by the extended Cauchy Integral
formula, we obtain
\eQb
6z &=& \dfrac{2!}{2\pi i} \int_{C} \dfrac{s^3 + 2s}{(s-z)^3i } ds,
\eQe
for $z$ at the interior of $C$. 
As $g(z) = \int_{C} \dfrac{s^3 +2s}{(s-z)^3} ds$, we have 
\eQb
g(z) &=& 6\pi i z ,
\eQe
for $z$ inside $C$. 
Now, if $z$ is outside of $C$, then $\dfrac{s^3 + 2s}
{s-z}$ is analytic at points interior to and on $C$. Hence, by
the Cauchy-Goursat Theorem, we have that
\eQb
\int_{C} \dfrac{s^3 + 2s}{(s-z)^3} ds &=& 0,
\eQe
for $z$ outside of $C$. 
Hence $g(z) = 0$ when $z$ is outside. $\qed$
\end{solution}

\bigskip
\begin{question}[8. Brown 170-7]
\end{question}
\begin{solution}
Let $C$ be the unit circle. As $e^{az}$ is entire, by the Cauchy Integral
formula, we obtain
\eQb
e^{az_0} &=& \dfrac{1}{2\pi i} \int_{C} \dfrac{e^{az}}{z - z_0} dz, 
\eQe
for $z_0$ inside $C$. By taking $z_0 = 0$, we get
\eQb
1 &=& \dfrac{1}{2 \pi i} \int_{C} \dfrac{e^{az}}{z} dz,
\eQe
which simplifies to 
\eQb
\int_{C} \dfrac{e^{az}}{z} dz &=& 2\pi i. 
\eQe
As $C$ is the unit circle, we have a polar parametrization as $z = e^{i\theta}$. Hence,
by substitution, we have
\eQb
\int_{C} \dfrac{e^{az}}{z} dz &=& 
\int_{-\pi}^{\pi} \dfrac{exp(ae^{i\theta})}{e^{i\theta}} ie^{i\theta}d\theta \\
&=& i\int_{-\pi}^{\pi} exp(ae^{i\theta}) d\theta \\
&=& i\int_{-\pi}^{\pi} e^{acos(\theta)}e^{iasin(\theta)} d\theta \\ 
&=& i\int_{-\pi}^{\pi} e^{acos(\theta)}(cos(asin(\theta)) + isin(asin(\theta)) d\theta \\
&=& -\int_{-\pi}^{\pi} sin(asin(\theta)d\theta + i\int_{-\pi}^{\pi} e^{acos(\theta)}
cos(asin(\theta)) d\theta
\eQe
As the above integral equals $2\pi i$, we have
\eQb
\int_{-\pi}^{\pi} e^{acos(\theta)}cos(asin(\theta)) d\theta &=& 2\pi,
\eQe
which, as the integrand is even, simplifies to
\eQb
\int_{0}^{\pi} e^{acos(\theta)}cos(asin(\theta)) d\theta &=& \pi,
\eQe
as desired. $\qed$
\end{solution}


\bigskip
\begin{question}[9. Brown 170-8]
\end{question}
\begin{solution}
The Legendre polynomials are defined by 
\eQb
P_n(z) = \dfrac{1}{2^{n+1}\pi i} \int_{C} 
\dfrac{(s^2 - 1)^n}{(s-z)^{n+1}} ds,
\eQe
for any simple closed contour surrounding $z$. 
For $z = -1$, and by having $C$ to be any arbitrary 
simple closed contour that surrounds $z = -1$, it follows that
\eQb
P_n(-1) &=& \dfrac{1}{2^{n+1}\pi i} \int_{C}
\dfrac{(s^2 -1)^n}{(s+1)^{n+1}} ds,
\eQe
which, by using the suggestion, simplifies to 
\eQb
P_n(z) &=& \dfrac{1}{2^{n+1} \pi i} \int_{C}
\dfrac{(s-1)^n}{s+1} ds.
\eQe
Since $(s-1)^n$ is entire, $(s-1)^n$ is analytic inside and 
on $C$. Hence, by the Cauchy Integral formula, we have
\eQb
(-2)^n 2 \pi i &=& \int_{C} 
\dfrac{(s-1)^n}{s+1} ds.
\eQe
Substituting the above equality into the simplified formula of
Legendre polynomials yields
\eQb
P_n(z) &=& \dfrac{(-2)^n2 \pi i}{2^{n+1}\pi i} \\
&=& (-1)^n,
\eQe
as desired. $\qed$

\end{solution}

\bigskip

\begin{question}[10. Brown 177-1]
\end{question}
\begin{solution}
Assume that $f(z)$ is entire, and that the harmonic function $u(x,y) = Re[f(z)]$
has an upper bound $u_0$. Observe that $g(z) = e^{f(z)}$ is entire, and 
\eQb
|e^{f(z)}| = |e^{u(x,y) + iv(x,y)}| = |e^{u(x,y)}| \leq e^{u_0},
\eQe
as $u_0$ is an upper bound for $u(x,y)$. Therefore, by the Liouville's theorem, we have
that $g$ is constant. Then, $g^{'}(z) = 0$ for all $z$. By the complex chain rule, we obtain
\eQb
g'(z) = f'(z) e^{f(z)}. 
\eQe
Since $e^{f(z)} \neq 0$, we have $f'(z) = 0$, and $f(z)$ is constant. 
Hence, the real part $u(x,y)$ is constant as well. $\qed$

\end{solution}


\end{document}
