\documentclass{article} % For LaTeX2e
\usepackage{nips14submit_e,times}
\usepackage{amsmath}
\usepackage{amsthm}
\usepackage{amssymb}
\usepackage{mathtools}
\usepackage{hyperref}
\usepackage{url}
\usepackage{algorithm}
\usepackage[noend]{algpseudocode}
%\documentstyle[nips14submit_09,times,art10]{article} % For LaTeX 2.09

\usepackage{graphicx}
\usepackage{caption}
\usepackage{subcaption}

\def\eQb#1\eQe{\begin{eqnarray*}#1\end{eqnarray*}}
\def\eQnb#1\eQne{\begin{eqnarray}#1\end{eqnarray}}
\providecommand{\e}[1]{\ensuremath{\times 10^{#1}}}
\providecommand{\pb}[0]{\pagebreak}

\newcommand{\E}{\mathrm{E}}
\newcommand{\Var}{\mathrm{Var}}
\newcommand{\Cov}{\mathrm{Cov}}

\def\Qb#1\Qe{\begin{question}#1\end{question}}
\def\Sb#1\Se{\begin{solution}#1\end{solution}}

\newenvironment{claim}[1]{\par\noindent\underline{Claim:}\space#1}{}
\newtheoremstyle{quest}{\topsep}{\topsep}{}{}{\bfseries}{}{ }{\thmname{#1}\thmnote{ #3}.}
\theoremstyle{quest}
\newtheorem*{definition}{Definition}
\newtheorem*{theorem}{Theorem}
\newtheorem*{lemma}{Lemma}
\newtheorem*{question}{Question}
\newtheorem*{preposition}{Preposition}
\newtheorem*{exercise}{Exercise}
\newtheorem*{challengeproblem}{Challenge Problem}
\newtheorem*{solution}{Solution}
\newtheorem*{remark}{Remark}
\usepackage{verbatimbox}
\usepackage{listings}
\title{Complex Analysis I: \\
Problem Set VIII}


\author{
Youngduck Choi \\
CILVR Lab \\
New York University\\
\texttt{yc1104@nyu.edu} \\
}


% The \author macro works with any number of authors. There are two commands
% used to separate the names and addresses of multiple authors: \And and \AND.
%
% Using \And between authors leaves it to \LaTeX{} to determine where to break
% the lines. Using \AND forces a linebreak at that point. So, if \LaTeX{}
% puts 3 of 4 authors names on the first line, and the last on the second
% line, try using \AND instead of \And before the third author name.

\newcommand{\fix}{\marginpar{FIX}}
\newcommand{\new}{\marginpar{NEW}}

\nipsfinalcopy % Uncomment for camera-ready version

\begin{document}


\maketitle

\begin{abstract}
This work contains the solutions to the problem set VIII
of Complex Analysis I 2015 at Courant Institute of Mathematical Sciences.
\end{abstract}

\bigskip

\begin{question}[1. 273-12]
\end{question}
\begin{solution}

\end{solution}

\bigskip

\begin{question}[1. 273-12]
\end{question}
\begin{solution}

\end{solution}

\bigskip

\begin{question}[287-1]
\end{question}
\begin{solution}
Observe that the following equality holds, by the linearity of integration: 
\eQb
\int_{0}^{2\pi} \dfrac{d\theta}{5 + 4\sin{\theta}} 
&=& \dfrac{1}{5} \int_{0}^{2\pi}
\dfrac{d\theta}{1+ \frac{4}{5}\sin{\theta}} \\
\eQe
From the example 1 from pg.285 in the section 92, it follows that
\eQb
\int_{0}^{2\pi} \dfrac{d\theta}{5 + 4\sin{\theta}} 
&=& \dfrac{1}{5} \int_{0}^{2\pi}
\dfrac{d\theta}{1+ \frac{4}{5}\sin{\theta}} \\
&=& \dfrac{1}{5} \dfrac{2\pi}{\sqrt{1-\frac{4}{5}^2}} \\
&=& \dfrac{1}{5} \dfrac{2\pi}{\frac{3}{5}} \\
&=& \dfrac{2}{3}\pi. 
\eQe
\hfill $\qed$
\end{solution}

\bigskip

\begin{question}[293-6]
\end{question}
\begin{solution}
\textbf{(a)}
Inside the circle $|z| = 1$, write 
\eQb
f(z) = -5z^4 \> \text{ and } \> g(z) = z^6 + z^3 - 2z.
\eQe
Then, observe that when $|z| = 1$,
\eQb
|f(z)| = 5|z|^4 = 5 \> \text{ and } \> |g(z)| \leq |z|^6 + |z|^3 + 2|z| = 4.
\eQe
The conditions of Rouche's theorem are thus satisfied. Consequently, 
since $f(z)$ has 4 zeroes, counting multiplicities, inside the circle $|z| 
=1$, $f(z) + g(z)$ has 4 zeroes. Therefore, the polynomial $z^6 -
5z^4 + z^3 - 2z$ has 4 zeroes inside the circle $|z| = 1$.
\hfill $\qed$

\smallskip

\textbf{(b)}
Inside the cirlce 

\smallskip
\textbf{(c)} 
Inside the circle $|z| = 1$, write
\eQb
f(z) = -4z^3 \> \text{ and } \> g(z) = z^7 + z - 1. 
\eQe
Then, observe that when $|z| = 1$, 
\eQb
|f(z)| = 4|z|^3 = 4 \> \text{ and } g(z) \leq |z|^7 + |z| - 1 = 1.
\eQe
The conditions of Rouche's theorem are thus satisfied. Consequently,
since $f(z)$ has 3 zeroes, counting multiplicities, inside the circle
$|z| = 1$, $f(z) + g(z)$ has 3 zeroes. Therefore, the polynomial 
$z^7 - 4z^3 + z - 1$ has 3 zeroes inside the circle $|z| = 1$.
\hfill $\qed$

\end{solution}

\bigskip

\begin{question}[293-8]
\end{question}
\begin{solution}

\end{solution}

\end{document}

