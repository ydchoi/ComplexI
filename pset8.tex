\documentclass{article} % For LaTeX2e
\usepackage{nips14submit_e,times}
\usepackage{amsmath}
\usepackage{amsthm}
\usepackage{amssymb}
\usepackage{mathtools}
\usepackage{hyperref}
\usepackage{url}
\usepackage{algorithm}
\usepackage[noend]{algpseudocode}
%\documentstyle[nips14submit_09,times,art10]{article} % For LaTeX 2.09

\usepackage{graphicx}
\usepackage{caption}
\usepackage{subcaption}

\def\eQb#1\eQe{\begin{eqnarray*}#1\end{eqnarray*}}
\def\eQnb#1\eQne{\begin{eqnarray}#1\end{eqnarray}}
\providecommand{\e}[1]{\ensuremath{\times 10^{#1}}}
\providecommand{\pb}[0]{\pagebreak}

\newcommand{\E}{\mathrm{E}}
\newcommand{\Var}{\mathrm{Var}}
\newcommand{\Cov}{\mathrm{Cov}}

\def\Qb#1\Qe{\begin{question}#1\end{question}}
\def\Sb#1\Se{\begin{solution}#1\end{solution}}

\newenvironment{claim}[1]{\par\noindent\underline{Claim:}\space#1}{}
\newtheoremstyle{quest}{\topsep}{\topsep}{}{}{\bfseries}{}{ }{\thmname{#1}\thmnote{ #3}.}
\theoremstyle{quest}
\newtheorem*{definition}{Definition}
\newtheorem*{theorem}{Theorem}
\newtheorem*{lemma}{Lemma}
\newtheorem*{question}{Question}
\newtheorem*{preposition}{Preposition}
\newtheorem*{exercise}{Exercise}
\newtheorem*{challengeproblem}{Challenge Problem}
\newtheorem*{solution}{Solution}
\newtheorem*{remark}{Remark}
\usepackage{verbatimbox}
\usepackage{listings}
\title{Complex Analysis I: \\
Problem Set VIII}


\author{
Youngduck Choi \\
CILVR Lab \\
New York University\\
\texttt{yc1104@nyu.edu} \\
}


% The \author macro works with any number of authors. There are two commands
% used to separate the names and addresses of multiple authors: \And and \AND.
%
% Using \And between authors leaves it to \LaTeX{} to determine where to break
% the lines. Using \AND forces a linebreak at that point. So, if \LaTeX{}
% puts 3 of 4 authors names on the first line, and the last on the second
% line, try using \AND instead of \And before the third author name.

\newcommand{\fix}{\marginpar{FIX}}
\newcommand{\new}{\marginpar{NEW}}

\nipsfinalcopy % Uncomment for camera-ready version

\begin{document}


\maketitle

\begin{abstract}
This work contains the solutions to the problem set VIII
of Complex Analysis I 2015 at Courant Institute of Mathematical Sciences.
\end{abstract}

\bigskip

\begin{question}[273-12]
\end{question}
\begin{solution}

\end{solution}

\bigskip

\begin{question}[278-1]
\end{question}
\begin{solution}
Let $f(z) = \dfrac{e^{iaz} - e^{ibz}}{z^2}$. On the indented
contour in figure 108, by Cauchy-Goursat theorem, we obtain
\eQb
\int_{L_1} f(z) dz + \int_{L_2} f(z) dz &=& -\int_{C_{\rho}} f(z) dz
- \int_{C_R} f(z) dz. \\
\eQe
Observe that on $L_1$ and $L_2$, we have the following parametric
representations:
\eQb
L_1 : z = re^{i0} = r (\rho \leq r \leq R) \> \text{ and } 
-L_2 : z = re^{i\pi} = -r (\rho \leq r \leq R).
\eQe
It follows that
\eQb
\int_{L_1} f(z) dz + \int_{L_2} f(z) dz &=& 
\int_{L_1} f(z)dz - \int_{-L_2} f(z) dz \\
&=& \int_{\rho}^{R} \dfrac{e^{iar} - e^{ibr}}{r^2} dr 
+ \int_{\rho}^{R} \dfrac{e^{-iar}- e^{-ibr}}{r^2} dr \\
&=& \int_{\rho}^{R} \dfrac{(e^{iar} + e^{-iar}) -
(e^{ibr} + e^{-ibr})}{r^2} dr \\
&=& 2\int_{\rho}^{R} \dfrac{\cos(ar) - \cos(br)}{r^2} dr.
\eQe
Therefore, we obtain
\end{solution}

\bigskip

\begin{question}[287-1]
\end{question}
\begin{solution}
Observe that the following equality holds, by the linearity of integration: 
\eQb
\int_{0}^{2\pi} \dfrac{d\theta}{5 + 4\sin{\theta}} 
&=& \dfrac{1}{5} \int_{0}^{2\pi}
\dfrac{d\theta}{1+ \frac{4}{5}\sin{\theta}} \\
\eQe
From the example 1 from pg.285 in the section 92, it follows that
\eQb
\int_{0}^{2\pi} \dfrac{d\theta}{5 + 4\sin{\theta}} 
&=& \dfrac{1}{5} \int_{0}^{2\pi}
\dfrac{d\theta}{1+ \frac{4}{5}\sin{\theta}} \\
&=& \dfrac{1}{5} \dfrac{2\pi}{\sqrt{1-\frac{4}{5}^2}} \\
&=& \dfrac{1}{5} \dfrac{2\pi}{\frac{3}{5}} \\
&=& \dfrac{2}{3}\pi. 
\eQe
\hfill $\qed$
\end{solution}

\bigskip

\begin{question}[287-6]
\end{question}
\begin{solution}
As $|\sin(-\pi+\theta)| = |\sin(\theta)|$, we have $\sin^{2n}(\theta) 
= \sin^{2n}(-\pi + \theta)$, and
$\int_{0}^{\pi}\sin^{2n}(\theta) d\theta = \int_{-\pi}^{0} \sin^{2n}(\theta)
d\theta$. Hence, it follows that
\eQb
\int_{0}^{\pi} \sin^{2n}(\theta) d\theta &=& \dfrac{1}{2}\int_{C}
\sin^{2n}(\theta), \\
\eQe 
where $C$ is the positively oriented unit circle $|z| = 1$. Substituting
$\dfrac{z - z^{-1}}{2i}$ for $\sin(\theta)$, using the binomial formula,
and linearity of integration, we obtain
\eQb
\dfrac{1}{2}\int_{C} \sin^{2n}(\theta) &=& 
\dfrac{1}{2}\int_{C} {\left( \dfrac{z - z^{-1}}{2i} \right)
}^{2n} \dfrac{dz}{iz} \\
&=& \dfrac{1}{2^{2n+1}(-1)^n i} \int_{C} \dfrac{(z-z^{-1})^{2n}}{z} dz \\
&=& \dfrac{1}{2^{2n+1}(-1)^{n} i} \int_{C} \sum_{k=0}^{n} 
\binom{2n}{k}z^{k}z^{2n-k}{(-z^{-1})}^{k}z^{-1} dz \\
&=& \dfrac{1}{2^{2n+1}(-1)^{n} i} \sum_{k=0}^{n} \binom{2n}{k}
(-1)^k \int_{C} z^{2n-2k-1}dz. 
\eQe
Observe that we only get non-zero integral value for $k = n$ case,
and $\int_{C} z^{-1} dz = 2\pi i$. Therefore, it follows that
\eQb
\int_{0}^{\pi} \sin^{2n}(\theta ) d\theta &=& 
\dfrac{1}{2^{2n+1}(-1)^{n} i} \dfrac{(2n)!}{n! n!} (-1)^n 2\pi i \\
&=& \dfrac{2n!}{2^{2n} (n!)^2}\pi,
\eQe
as desired. \hfill $\qed$

\end{solution}

\bigskip

\begin{question}[293-6]
\end{question}
\begin{solution}
\textbf{(a)}
Inside the circle $|z| = 1$, write 
\eQb
f(z) = -5z^4 \> \text{ and } \> g(z) = z^6 + z^3 - 2z.
\eQe
Then, observe that when $|z| = 1$,
\eQb
|f(z)| = 5|z|^4 = 5 \> \text{ and } \> |g(z)| \leq |z|^6 + |z|^3 + 2|z| = 4.
\eQe
The conditions of Rouche's theorem are thus satisfied. Consequently, 
since $f(z)$ has 4 zeroes, counting multiplicities, inside the circle $|z| 
=1$, $f(z) + g(z)$ has 4 zeroes, inside the circle $|z| = 1$.
Therefore, the polynomial $z^6 -
5z^4 + z^3 - 2z$ has 4 zeroes, inside the circle $|z| = 1$.
\hfill $\qed$

\smallskip

\textbf{(b)}
Inside the circle $|z| = 1$, write
\eQb
f(z) = 9 \> \text{ and } \> g(z) = 2z^4 - 2z^3 + 2z^2 - 2z.
\eQe 
Then, observe that when $|z| = 1$,
\eQb
|f(z)| = 9 \> \text{ and } \> |g(z)| = 2|z|^4 + 2|z|^3 + 2|z|^2 + 2|z| = 8.
\eQe
The conditions of Rouche's theorem are thus satisfied. Consequently,
since $f(z)$ has 0 zero, counting multiplicities, inside the circle $|z| = 1$,
$f(z) + g(z)$ has 0 zero, inside the circle $|z| = 1$. Therefore, 
the polynomial $2z^4 - 2z^3 + 2z^2 - 2z + 9$ has 0 zero, inside
the circle $|z| = 1$.
\hfill $\qed$

\smallskip
\textbf{(c)} 
Inside the circle $|z| = 1$, write
\eQb
f(z) = -4z^3 \> \text{ and } \> g(z) = z^7 + z - 1. 
\eQe
Then, observe that when $|z| = 1$, 
\eQb
|f(z)| = 4|z|^3 = 4 \> \text{ and } |g(z)| \leq |z|^7 + |z| - 1 = 1.
\eQe
The conditions of Rouche's theorem are thus satisfied. Consequently,
since $f(z)$ has 3 zeroes, counting multiplicities, inside the circle
$|z| = 1$, $f(z) + g(z)$ has 3 zeroes. Therefore, the polynomial 
$z^7 - 4z^3 + z - 1$ has 3 zeroes inside the circle $|z| = 1$.
\hfill $\qed$

\end{solution}

\bigskip

\begin{question}[293-8]
\end{question}
\begin{solution}
Inside the circle $|z| = 2$, write
\eQb
f(z) = 2z^5 \> \text{ and } \> g(z) = 6z^2 + z + 1.
\eQe
Then, observe that when $|z| = 2$,
\eQb
|f(z)| = 2|z|^5 = 64 \> \text{ and } |g(z)| \leq 6|z|^2 + |z| + |1| = 8.
\eQe
The conditions of Rouche's theorem are thus satisfied. Consequently, since
$f(z)$ has 5 zeroes, counting multiplicities, inside the circle $|z| = 2$,
$f(z) + g(z)$ has 5 zeroes. On the other hand, inside the circle $|z| = 1$,
write
\eQb
f(z) = -6z^2 \> \text{ and } \> g(z) = 2z^5 + z + 1.
\eQe
Then, observe that when $|z| = 1$,
\eQb
|f(z)| = 6|z|^2 = 6 \> \text{ and } |g(z)| \leq 2|z|^5 + |z| + 1 = 4.
\eQe
The conditions of Rouche's theorem are thus satisfied. Consequently, since
$f(z)$ has 2 zeroes, counting multiplicities, inside the circle $|z| = 1 $,
$f(z) + g(z)$ has 2 zeroes. Therefore, we have shown that
in the annulus $1 \leq |z| \leq 2$, we have $5 - 2 = 3$ zeroes. 
\hfill $\qed$
\end{solution}

\end{document}

