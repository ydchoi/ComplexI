\documentclass{article} % For LaTeX2e
\usepackage{nips14submit_e,times}
\usepackage{amsmath}
\usepackage{amsthm}
\usepackage{amssymb}
\usepackage{mathtools}
\usepackage{hyperref}
\usepackage{url}
\usepackage{algorithm}
\usepackage[noend]{algpseudocode}
%\documentstyle[nips14submit_09,times,art10]{article} % For LaTeX 2.09

\usepackage{graphicx}
\usepackage{caption}
\usepackage{subcaption}

\def\eQb#1\eQe{\begin{eqnarray*}#1\end{eqnarray*}}
\def\eQnb#1\eQne{\begin{eqnarray}#1\end{eqnarray}}
\providecommand{\e}[1]{\ensuremath{\times 10^{#1}}}
\providecommand{\pb}[0]{\pagebreak}

\newcommand{\E}{\mathrm{E}}
\newcommand{\Var}{\mathrm{Var}}
\newcommand{\Cov}{\mathrm{Cov}}

\def\Qb#1\Qe{\begin{question}#1\end{question}}
\def\Sb#1\Se{\begin{solution}#1\end{solution}}

\newenvironment{claim}[1]{\par\noindent\underline{Claim:}\space#1}{}
\newtheoremstyle{quest}{\topsep}{\topsep}{}{}{\bfseries}{}{ }{\thmname{#1}\thmnote{ #3}.}
\theoremstyle{quest}
\newtheorem*{definition}{Definition}
\newtheorem*{theorem}{Theorem}
\newtheorem*{lemma}{Lemma}
\newtheorem*{question}{Question}
\newtheorem*{preposition}{Preposition}
\newtheorem*{exercise}{Exercise}
\newtheorem*{challengeproblem}{Challenge Problem}
\newtheorem*{solution}{Solution}
\newtheorem*{remark}{Remark}
\usepackage{verbatimbox}
\usepackage{listings}
\title{Complex Analysis I: \\
Problem Set VII}


\author{
Youngduck Choi \\
CILVR Lab \\
New York University\\
\texttt{yc1104@nyu.edu} \\
}


% The \author macro works with any number of authors. There are two commands
% used to separate the names and addresses of multiple authors: \And and \AND.
%
% Using \And between authors leaves it to \LaTeX{} to determine where to break
% the lines. Using \AND forces a linebreak at that point. So, if \LaTeX{}
% puts 3 of 4 authors names on the first line, and the last on the second
% line, try using \AND instead of \And before the third author name.

\newcommand{\fix}{\marginpar{FIX}}
\newcommand{\new}{\marginpar{NEW}}

\nipsfinalcopy % Uncomment for camera-ready version

\begin{document}


\maketitle

\begin{abstract}
This work contains the solutions to the problem set VII
of Complex Analysis I 2015 at Courant Institute of Mathematical Sciences.
\end{abstract}

\bigskip

\begin{question}[1. 247-7]
\end{question}
\begin{solution}
\textbf{(a)} Observe that
\eQb
\dfrac{1}{z^2}f(\dfrac{1}{z}) &=& \dfrac{(3+2z)^2}{z(1-z)(2+5z)}.
\eQe
As $\dfrac{1}{z^2}f(\dfrac{1}{z})$ has a simple pole at $z = 0$, we have
\eQb
\int_{C} \dfrac{(3+2z)^2}{z(z-1)(2z+5)}dz &=& 2\pi i \cdot 
\text{Res}_{z=0} [\dfrac{1}{z^2}f(\dfrac{1}{z})] \\
&=& 2\pi i \cdot \dfrac{9}{2}  
= 9\pi i.
\eQe

\smallskip

\textbf{(b)} Observe that
\eQb
\dfrac{1}{z^2}f(\dfrac{1}{z}) = \dfrac{e^z}{z^2(1+z^3)}.  
\eQe
As $\dfrac{1}{z^2}f(\dfrac{1}{z})$ has a pole of order $2$ at $z = 0$,
we have
\eQb
\int_{C} \dfrac{z^3 e^{\frac{1}{z}}}{1+z^3} dz &=& 2\pi i 
\cdot \text{Res}_{z = 0}[\dfrac{1}{z^2}f(\dfrac{1}{z})],
\eQe
where $\phi(z) = \dfrac{e^z}{1+z^3}$. We have
\eQb
\phi^{'}(z) &=& \dfrac{(1+z^3)e^z - e^z 3z^2}{(1+z^3)^2}. 
\eQe
By substituting $z = 0$, we see that $\phi^{'}(0) = 1$, which is the residue
at $z = 0$. It follows that 
\eQb
\int_{C} \dfrac{z^3 e^{\frac{1}{z}}}{1+z^3} dz &=& 2\pi i. 
\eQe
\qed
\end{solution}

\bigskip

\begin{question}[2. 254-5]
\end{question}
\begin{solution}
\textbf{(a)}
The given integral can be written as
\eQb
\int_{C} \tan(z) dz &=& \int_{C} \dfrac{p(z)}{q(z)} dz \\ 
&=& \int_{C} \dfrac{\sin(z)}{\cos(z)} dz.
\eQe
As the zeros of $\cos(z)$ are $z = \dfrac{\pi}{2} + n\pi$ and $C$ is
the positively oriented circle $|z| = 2$, there are two isolated singularities
of $\tan(z)$ interior to $C$, namely $z = \pm \dfrac{\pi}{2}$. It follows
that
\eQb
\text{Res}_{z = \frac{\pi}{2}} \tan(z) &=& 
\dfrac{p(\frac{\pi}{2})}{q^{'}(\frac{\pi}{2})} 
= \dfrac{\sin(\frac{\pi}{2})}{-\sin(\frac{\pi}{2})} = -1, \\ 
\text{Res}_{z = \frac{-\pi}{2}} \tan(z) &=& 
\dfrac{p(\frac{-\pi}{2})}{q^{'}(\frac{-\pi}{2})} 
= \dfrac{\sin(\frac{-\pi}{2})}{-\sin(\frac{-\pi}{2})} = -1. \\
\eQe
Consequently, by the residue theorem, we have
\eQb
\int_{C} \tan(z) dz &=& -4\pi i.
\eQe

\smallskip

\textbf{(b)} 
We wish to evaluate the integral $\int_{C} \dfrac{dz}{\sinh(2z)}$. 
As $\sinh(z) = 0$ for $\dfrac{n\pi i}{2}$, we can conclude that
the isolated singularities of the integrand happens at $z = 0$ and
$z = \pm \dfrac{\pi i}{2}$. It follows that
\eQb
\text{Res}_{z=0} \dfrac{1}{\sinh(2z)} &=& \dfrac{1}{2\cosh(0)} 
= \dfrac{1}{2} \\
\text{Res}_{z=\frac{\pi i}{2}} 
\dfrac{1}{\sinh(2z)} &=& \dfrac{1}{2\cosh(\pi i)} 
= -\dfrac{1}{2} \\
\text{Res}_{z=-\frac{\pi i}{2}} 
\dfrac{1}{\sinh(2z)} &=& \dfrac{1}{2\cosh(-\pi i)} 
= -\dfrac{1}{2} \\
\eQe
Consequently, by the residue theorem, we have
\eQb
\int_{C} \dfrac{1}{\sinh(2z)} dz = -\pi i.
\eQe
\qed
\end{solution}

\bigskip

\begin{question}[3. 254-6]
\end{question}
\begin{solution}
Observe that interior to $C_N$, the function $\dfrac{1}{z^2\sin(z)}$ 
has singularities at $z = 0$ and $z = \pm n\pi$ for $n = 1,2,...,N$.
The residue at $z=0$ can be found by finding the $\dfrac{1}{z}$ coefficient
of $\dfrac{1}{z^2\sin(z)}$, which was found to be $\dfrac{1}{6}$ in a 
previous homework problem. For the residues at $z = \pm n \pi$ for 
$n = 1,2,...,N$, we have
\eQb
\dfrac{1}{z^2\sin(z)} = \dfrac{p(z)}{q(z)} \> \text{where} \>
p(z) = 1 , q(z) = z^2 \sin (z).
\eQe
As $q^{'}(z) = z^2cos(z)$, we have for $z = \pm n\pi$, we have
$q^{'}(\pm n\pi) = (-1)^{n}n^{2}\pi^{2}$. By the residue theorem, it follows
that
\eQb
\int_{C_N} \dfrac{1}{z^2\sin(z)}dz &=& 2\pi i [\dfrac{1}{6} + 
2 \sum_{n=1}^{N} \dfrac{(-1)^n}{n^2 \pi^2} ],
\eQe 
which can be re-written as
\eQb
\sum_{n=1}^{N} \dfrac{(-1)^{n+1}}{n^2} = \dfrac{\pi^2}{12} 
- \dfrac{\pi}{4i} \int_{C_N} \dfrac{1}{z^2\sin(z)}dz.
\eQe
We have that as $N \to \infty$, the integral goes to $0$. Therefore,
it follows that
\eQb
\sum_{n=1}^{\infty} \dfrac{(-1)^{n+1}}{n^2} = \dfrac{\pi^2}{12}. 
\eQe
$\qed$

\end{solution}

\bigskip

\begin{question}[254-8]
\end{question}
\begin{solution}

\end{solution}

\begin{question}[264.2]
\end{question}
\begin{solution}
Consider the function $f(z) = \dfrac{1}{(z^2+1)^2}$ and the simple,
closed, and positively oriented  
contour of a half-circle above the real-axis, centered around the
origin with the radius $R > 1$. It follows that
\eQb
\int_{-R}^{R} \dfrac{1}{(x^2+1)^2} dx + \int_{C_R} \dfrac{1}{(z^2+1)^2}dz
&=& 2\pi B,
\eQe 
where $C_R$ denotes the contour of the curve part of the half-circle,
and $B$ is the residue of the complex integrand at $z=i$. For computing
the residue, we have
\eQb
\dfrac{1}{(z^2+1)} &=& \dfrac{\phi(z)}{(z-i)^2},
\eQe 
where $\phi(z) = \dfrac{1}{(z+i)^2}$. It follows that $B = \phi^{'}(i) 
= \dfrac{1}{4i}$. Therefore, we obtain
\eQb
\int_{-R}^{R} \dfrac{dx}{(x^2+1)^2} = \dfrac{\pi}{2} - 
\int_{C_R} \dfrac{1}{(z^2+1)^2} dz.
\eQe
For $z$ on $C_R$, we have $|z^2+1| \geq R^2 - 1$. Therefore, we have
\eQb
\left| \int_{C_R} \dfrac{1}{(z^2+1)^2} dz \right| 
\leq \dfrac{\pi R}{(R^2 -1)^2}.
\eQe 
Consequently, we have as $R \to \infty, \dfrac{\pi R}{(R^2-1)^2} \to 0$. 
It follows that 
\eQb
\int_{-\infty}^{\infty} \dfrac{1}{(x^2+1)^2}dx &=& \dfrac{\pi}{2}. \\ 
\eQe 
As the integrand is an even function, we obtain that
\eQb
\int_{0}^{\infty} \dfrac{1}{(x^2+1)^2} dx = \dfrac{\pi}{4}. \\
\eQe
\qed
\end{solution}

\begin{question}[264.6]
\end{question}
\begin{solution}
Consider the function $f(z) = \dfrac{z^2}{(z^2+9)(z^+4)^2}$ and the simple,
closed, and positively oriented  
contour of a half-circle above the real-axis, centered around the
origin with the radius $R > 3$. It follows that
\eQb
\int_{-R}^{R} \dfrac{x^2}{(x^2+9)(z^2+4)^2} dx +
\int_{C_R} \dfrac{z^2}{(z^2+9)(z^2+4)^2}dz
&=& 2\pi (B_1 + B_2).
\eQe 
where $C_R$ denotes the contour of the curve part of the half-circle,
and $B_1$ and $B_2$ are the residues
 of the complex integrand at $z=3i$ and $z=zi$ respectively. 
First, observe that
\eQb
B_1 &=& \text{Res}_{z=3i} \dfrac{z^2}{(z^2+9)(z^2+4)^2} \\
&=& \dfrac{z^2}{(z+3i)(z^2+4)^2}\|_{z=3i} = -\dfrac{3}{50i}.
\eQe
For computing the $B_2$ term, observe that
\eQb
\dfrac{z^2}{(z^2+9)(z^2+4)^2} &=& \dfrac{\phi(z)}{(z-2i)^2}, \\
&\text{where}& \\
\phi(z) &=& \dfrac{z^2}{(z^2+9)(z+2i)^2}.
\eQe
It follows that 
\eQb
B_2 &=& \phi^{'}(2i) = \dfrac{13}{200i}.
\eQe
Hence, we have
\eQb
\int_{-R}^{R} \dfrac{x^2}{(x^2+9)(z^2+4)^2} dx 
&=& \dfrac{\pi}{100} - \int_{C_R} \dfrac{z^2}{(z^2+9)(z^2+4)^2}dz .
\eQe
Observe that
\eQb
\left| \int_{C_R} \dfrac{z^2}{(z^2+9)(z^2+4)^2} \right| &\leq& 
\dfrac{R^2}{(R^2-9)(R^2-4)^2}\pi R
\eQe
The RHS of the inequality goes to $0$ as $R \to \infty$. With the fact
that the integrand is even, we finally obtain
\eQb
\int_{0}^{\infty} \dfrac{x^2}{(x^2+9)(x^2+4)^2} dx = \dfrac{\pi}{200}.
\eQe
\qed
\end{solution}

\bigskip

\begin{question}[264.8]
\end{question}
\begin{solution}
Consider the function $f(z) = \dfrac{z}{(z^2+1)(x^2+2x+2)}$, and the
simple, closed, and positively oriented contour of a semi-circle with the
radius strictly larger than $\sqrt(2)$ and the center of the circle at the
origin. It follows that the isolated singularities are at $z = i$ and
$z = -1 +i$. By the residue theorem, it follows that
\eQb
\int_{-R}^{R} f(x)dx + \int_{C_R} f(z) dz &=& 2\pi i(B_0 + B_1),
\eQe
where $B_0$ and $B_1$ are the residues at the singularities respectively.
It follows that
\eQb
B_0 &=& \text{Res}_{z = i}f(z) \\
&=& [\dfrac{z}{(z+i)(z^2 + 2z + 2)}]_{z=i} = \dfrac{1}{10} - \dfrac{1}{5}i \\
B_1 &=& \text{Res}_{z= -1 +i} f(z) \\
&=& [\dfrac{z}{(z^2+1)(z+1+i})]_{z = -1 +i} = -\dfrac{1}{10} -\dfrac{3}{10}i. 
\\
\eQe
By the residue theorem, we obtain
\eQb
\int_{-R}^{R} \dfrac{x}{(x^2+1)(x^2+2x+2)} dx
&=& -\dfrac{\pi}{5} - \int_{C_N} \dfrac{z}{(z^2+1)(z^2+2z+2)}. \\
\eQe
Observe that by the RL-inequality,
\eQb
\left| \int_{C_R} \dfrac{z}{(z^2+1)(z^2+2z+2)} \right| 
&\leq& \dfrac{\pi R^2}{(R^2-1)(R-\sqrt{2})^2}.
\eQe
Hence, the integral over the $C_N$ contour tends to $0$, has $R \to \infty$.
Therefore, we have
\eQb
\lim_{R \to \infty} \int_{-R}^{R} \dfrac{x}{(x^2+1)(x^2+2x+2)} 
&=& -\dfrac{\pi}{5}.
\eQe
\qed

\end{solution}

\bigskip

\begin{question}[264.9]
\end{question}
\begin{solution}
Observe that interior to 
\end{solution}


\end{document}
