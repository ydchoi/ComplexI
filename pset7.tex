\documentclass{article} % For LaTeX2e
\usepackage{nips14submit_e,times}
\usepackage{amsmath}
\usepackage{amsthm}
\usepackage{amssymb}
\usepackage{mathtools}
\usepackage{hyperref}
\usepackage{url}
\usepackage{algorithm}
\usepackage[noend]{algpseudocode}
%\documentstyle[nips14submit_09,times,art10]{article} % For LaTeX 2.09

\usepackage{graphicx}
\usepackage{caption}
\usepackage{subcaption}

\def\eQb#1\eQe{\begin{eqnarray*}#1\end{eqnarray*}}
\def\eQnb#1\eQne{\begin{eqnarray}#1\end{eqnarray}}
\providecommand{\e}[1]{\ensuremath{\times 10^{#1}}}
\providecommand{\pb}[0]{\pagebreak}

\newcommand{\E}{\mathrm{E}}
\newcommand{\Var}{\mathrm{Var}}
\newcommand{\Cov}{\mathrm{Cov}}

\def\Qb#1\Qe{\begin{question}#1\end{question}}
\def\Sb#1\Se{\begin{solution}#1\end{solution}}

\newenvironment{claim}[1]{\par\noindent\underline{Claim:}\space#1}{}
\newtheoremstyle{quest}{\topsep}{\topsep}{}{}{\bfseries}{}{ }{\thmname{#1}\thmnote{ #3}.}
\theoremstyle{quest}
\newtheorem*{definition}{Definition}
\newtheorem*{theorem}{Theorem}
\newtheorem*{lemma}{Lemma}
\newtheorem*{question}{Question}
\newtheorem*{preposition}{Preposition}
\newtheorem*{exercise}{Exercise}
\newtheorem*{challengeproblem}{Challenge Problem}
\newtheorem*{solution}{Solution}
\newtheorem*{remark}{Remark}
\usepackage{verbatimbox}
\usepackage{listings}
\title{Complex Analysis I: \\
Problem Set VII}


\author{
Youngduck Choi \\
CILVR Lab \\
New York University\\
\texttt{yc1104@nyu.edu} \\
}


% The \author macro works with any number of authors. There are two commands
% used to separate the names and addresses of multiple authors: \And and \AND.
%
% Using \And between authors leaves it to \LaTeX{} to determine where to break
% the lines. Using \AND forces a linebreak at that point. So, if \LaTeX{}
% puts 3 of 4 authors names on the first line, and the last on the second
% line, try using \AND instead of \And before the third author name.

\newcommand{\fix}{\marginpar{FIX}}
\newcommand{\new}{\marginpar{NEW}}

\nipsfinalcopy % Uncomment for camera-ready version

\begin{document}


\maketitle

\begin{abstract}
This work contains the solutions to the problem set VII
of Complex Analysis I 2015 at Courant Institute of Mathematical Sciences.
\end{abstract}

\bigskip

\begin{question}[1. 247-7]
\end{question}
\begin{solution}
\textbf{(a)} Observe that
\eQb
\dfrac{1}{z^2}f(\dfrac{1}{z}) &=& \dfrac{(3+2z)^2}{z(1-z)(2+5z)}.
\eQe
As $\dfrac{1}{z^2}f(\dfrac{1}{z})$ has a simple pole at $z = 0$, we have
\eQb
\int_{C} \dfrac{(3+2z)^2}{z(z-1)(2z+5)}dz &=& 2\pi i \cdot 
\text{Res}_{z=0} [\dfrac{1}{z^2}f(\dfrac{1}{z})] \\
&=& 2\pi i \cdot \dfrac{9}{2}  
= 9\pi i.
\eQe

\smallskip

\textbf{(b)} Observe that
\eQb
\dfrac{1}{z^2}f(\dfrac{1}{z}) = \dfrac{e^z}{z^2(1+z^3)}.  
\eQe
As $\dfrac{1}{z^2}f(\dfrac{1}{z})$ has a pole of order $2$ at $z = 0$,
we have
\eQb
\int_{C} \dfrac{z^3 e^{\frac{1}{z}}}{1+z^3} dz &=& 2\pi i 
\cdot \text{Res}_{z = 0}[\dfrac{1}{z^2}f(\dfrac{1}{z})],
\eQe
where $\phi(z) = \dfrac{e^z}{1+z^3}$. We have
\eQb
\phi^{'}(z) &=& \dfrac{(1+z^3)e^z - e^z 3z^2}{(1+z^3)^2}. 
\eQe
By substituting $z = 0$, we see that $\phi^{'}(0) = 1$, which is the residue
at $z = 0$. It follows that 
\eQb
\int_{C} \dfrac{z^3 e^{\frac{1}{z}}}{1+z^3} dz &=& 2\pi i. 
\eQe
\qed
\end{solution}

\bigskip

\begin{question}[2. 254-5]
\end{question}
\begin{solution}
\textbf{(a)}
The given integral can be written as
\eQb
\int_{C} \tan(z) dz &=& \int_{C} \dfrac{p(z)}{q(z)} dz \\ 
&=& \int_{C} \dfrac{\sin(z)}{\cos(z)} dz.
\eQe
As the zeros of $\cos(z)$ are $z = \dfrac{\pi}{2} + n\pi$ and $C$ is
the positively oriented circle $|z| = 2$, there are two isolated singularities
of $\tan(z)$ interior to $C$, namely $z = \pm \dfrac{\pi}{2}$. It follows
that
\eQb
\text{Res}_{z = \frac{\pi}{2}} \tan(z) &=& 
\dfrac{p(\frac{\pi}{2})}{q^{'}(\frac{\pi}{2})} 
= \dfrac{\sin(\frac{\pi}{2})}{-\sin(\frac{\pi}{2})} = -1, \\ 
\text{Res}_{z = \frac{-\pi}{2}} \tan(z) &=& 
\dfrac{p(\frac{-\pi}{2})}{q^{'}(\frac{-\pi}{2})} 
= \dfrac{\sin(\frac{-\pi}{2})}{-\sin(\frac{-\pi}{2})} = -1. \\
\eQe
Consequently, by the residue theorem, we have
\eQb
\int_{C} \tan(z) dz &=& -4\pi i.
\eQe

\smallskip

\textbf{(b)} 
We wish to evaluate the integral $\int_{C} \dfrac{dz}{\sinh(2z)}$. 
As $\sinh(z) = 0$ for $\dfrac{n\pi i}{2}$, we can conclude that
the isolated singularities of the integrand happens at $z = 0$ and
$z = \pm \dfrac{\pi i}{2}$. It follows that
\eQb
\text{Res}_{z=0} \dfrac{1}{\sinh(2z)} &=& \dfrac{1}{2\cosh(0)} 
= \dfrac{1}{2} \\
\text{Res}_{z=\frac{\pi i}{2}} 
\dfrac{1}{\sinh(2z)} &=& \dfrac{1}{2\cosh(\pi i)} 
= -\dfrac{1}{2} \\
\text{Res}_{z=-\frac{\pi i}{2}} 
\dfrac{1}{\sinh(2z)} &=& \dfrac{1}{2\cosh(-\pi i)} 
= -\dfrac{1}{2} \\
\eQe
Consequently, by the residue theorem, we have
\eQb
\int_{C} \dfrac{1}{\sinh(2z)} dz = -\pi i.
\eQe
\qed
\end{solution}

\bigskip

\begin{question}[3. 254-6]
\end{question}
\begin{solution}
\end{solution}

\bigskip

\begin{question}[1. 237-2]
\end{question}
\begin{solution}
\end{solution}
\end{document}
