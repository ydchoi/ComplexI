\documentclass{article} % For LaTeX2e
\usepackage{nips14submit_e,times}
\usepackage{amsmath}
\usepackage{amsthm}
\usepackage{amssymb}
\usepackage{mathtools}
\usepackage{hyperref}
\usepackage{url}
\usepackage{algorithm}
\usepackage[noend]{algpseudocode}
%\documentstyle[nips14submit_09,times,art10]{article} % For LaTeX 2.09

\usepackage{graphicx}
\usepackage{caption}
\usepackage{subcaption}

\def\eQb#1\eQe{\begin{eqnarray*}#1\end{eqnarray*}}
\def\eQnb#1\eQne{\begin{eqnarray}#1\end{eqnarray}}
\providecommand{\e}[1]{\ensuremath{\times 10^{#1}}}
\providecommand{\pb}[0]{\pagebreak}

\newcommand{\E}{\mathrm{E}}
\newcommand{\Var}{\mathrm{Var}}
\newcommand{\Cov}{\mathrm{Cov}}

\def\Qb#1\Qe{\begin{question}#1\end{question}}
\def\Sb#1\Se{\begin{solution}#1\end{solution}}

\newenvironment{claim}[1]{\par\noindent\underline{Claim:}\space#1}{}
\newtheoremstyle{quest}{\topsep}{\topsep}{}{}{\bfseries}{}{ }{\thmname{#1}\thmnote{ #3}.}
\theoremstyle{quest}
\newtheorem*{definition}{Definition}
\newtheorem*{theorem}{Theorem}
\newtheorem*{lemma}{Lemma}
\newtheorem*{question}{Question}
\newtheorem*{preposition}{Preposition}
\newtheorem*{exercise}{Exercise}
\newtheorem*{challengeproblem}{Challenge Problem}
\newtheorem*{solution}{Solution}
\newtheorem*{remark}{Remark}
\usepackage{verbatimbox}
\usepackage{listings}
\title{Complex Analysis I: \\
Problem Set V}


\author{
Youngduck Choi \\
CILVR Lab \\
New York University\\
\texttt{yc1104@nyu.edu} \\
}


% The \author macro works with any number of authors. There are two commands
% used to separate the names and addresses of multiple authors: \And and \AND.
%
% Using \And between authors leaves it to \LaTeX{} to determine where to break
% the lines. Using \AND forces a linebreak at that point. So, if \LaTeX{}
% puts 3 of 4 authors names on the first line, and the last on the second
% line, try using \AND instead of \And before the third author name.

\newcommand{\fix}{\marginpar{FIX}}
\newcommand{\new}{\marginpar{NEW}}

\nipsfinalcopy % Uncomment for camera-ready version

\begin{document}


\maketitle

\begin{abstract}
This work contains the solutions to the problem set V
of Complex Analysis I 2015 at Courant Institute of Mathematical Sciences.
\end{abstract}

\bigskip

\begin{question}[1. 177.2]
\end{question}
\begin{solution}
Let $f$ be continuous on a closed bounded region $R$, and let it be
analytic and not constant throughout the interior of $R$. Assume 
that $f(z) \neq 0$ for $z \in R$. Let $g$ be a function on $R$,
defined by $g(z) = \dfrac{1}{f(z)}$ for $z \in R$. From the $g(z) = 
\dfrac{1}{f(z)}$ relation, we can deduce that
$g$ is also continuous, analytic and not constant throughout the interior
of $R$. Then, by the given corollary of the maximum modulus principle,
we have that the maximum value of $|g(z)|$ in $R$, which is always 
reached, occurs somewhere on the boundary of $R$ and never in the interior.
Observe that $|g(z)| = |\dfrac{1}{f(z)}| = \dfrac{1}{|f(z)|}$. Since
a modulus is strictly positive in this case, we have that maximum value
of $|g(z)|$ corresponds to the minimum value of $|f(z)|$. 
In other words,
the $z^*$, which is $\text{argmax} |g(z)|$ and lies on the boundary, is also
the $\text{argmin} |f(z)|$. Consequently, we have shown that a minimum
value is reached, and it occurs in the boundary of $R$ and never in
the interior. $\qed$ 
\end{solution}

\bigskip

\begin{question}[2. 177.4]
\end{question}
\begin{solution}

\end{solution}

\bigskip

\begin{question}[1.177.5]
\end{question}
\begin{solution}
\end{solution}

\bigskip

\begin{question}[4. 195.3]
\end{question}
\begin{solution}
We wish to find the Maclaurin series expansion of the function
\eQb
f(z) &=& \dfrac{z}{z^4 + 4} = \dfrac{z}{4} \cdot 
\dfrac{1}{1 + (z^4/4)}.
\eQe
As the given function $f$ is analytic throughout a disk
$D = \{ z \> | \> |z - z_0| < R_0 \}$, 
where $z_0 = 0$ and $R_0 = \sqrt{2}$,
by the Taylor's theorem, we have a Maclaurin series
expansion for $f$ as
\eQb
f(z) &=& \sum_{n=0}^{\infty} \dfrac{f^{(n)}(0)}{n!}z^n, 
\eQe
for $|z| < \sqrt{2}$. For $f$, the nth derivative can be computed,
using the chain rule and product rule of differentiation,
as follows:
\eQb
f^{(1)}(z) &=& \dfrac{1}{4} \cdot \dfrac{1}{1+(z^4/4))} + 
\dfrac{z}{4}\cdot \dfrac{-z^3}{1+(z^4/4)} \\
 &=& \dfrac{1}{4}\left( \dfrac{1-z^4}{1+(z^4/4)} \right) \\
f^{(2)}(z) &=& \dfrac{1}{4}
\eQe
 
\end{solution}

\bigskip

\begin{question}[5. 195.6]
\end{question}
\begin{solution}
From pg.193, we are given a Maclaurin series expansion of $\sinh$ as
\eQb
\sinh (z) &=& \sum_{i=0}^{\infty} \dfrac{z^{2n+1}}{(2n+1)!},
\eQe
for all $z \in \mathbb{C}$.
From pg.196, we are given that $\sinh(z + \pi i) = -\sinh(z)$ and
$\sinh$ is $2\pi i$ periodic. Consequently, we have that
$\sinh(z - \pi i) = -\sinh(z)$. Substituting the equality into
the above Taylor series expansion yields
\eQb
\sinh(z -\pi i) &=& \sum_{i=0}^{\infty} \dfrac{(z-\pi i)^{2n+1}}{(2n+1)!} \\
\sinh(z) &=& -\sum_{i=0}^{\infty} \dfrac{(z-\pi i)^{2n+1}}{(2n+1)!}, \\
\eQe
for any $z \in \mathbb{C}$.
as desired. $\qed$
\end{solution}

\bigskip

\begin{question}[6. 195.11]
\end{question}
\begin{solution}
\end{solution}

\bigskip

\end{document}
