\documentclass{article} % For LaTeX2e
\usepackage{nips14submit_e,times}
\usepackage{amsmath}
\usepackage{amsthm}
\usepackage{amssymb}
\usepackage{mathtools}
\usepackage{hyperref}
\usepackage{url}
\usepackage{algorithm}
\usepackage[noend]{algpseudocode}
%\documentstyle[nips14submit_09,times,art10]{article} % For LaTeX 2.09

\usepackage{graphicx}
\usepackage{caption}
\usepackage{subcaption}

\def\eQb#1\eQe{\begin{eqnarray*}#1\end{eqnarray*}}
\def\eQnb#1\eQne{\begin{eqnarray}#1\end{eqnarray}}
\providecommand{\e}[1]{\ensuremath{\times 10^{#1}}}
\providecommand{\pb}[0]{\pagebreak}

\newcommand{\E}{\mathrm{E}}
\newcommand{\Var}{\mathrm{Var}}
\newcommand{\Cov}{\mathrm{Cov}}

\def\Qb#1\Qe{\begin{question}#1\end{question}}
\def\Sb#1\Se{\begin{solution}#1\end{solution}}

\newenvironment{claim}[1]{\par\noindent\underline{Claim:}\space#1}{}
\newtheoremstyle{quest}{\topsep}{\topsep}{}{}{\bfseries}{}{ }{\thmname{#1}\thmnote{ #3}.}
\theoremstyle{quest}
\newtheorem*{definition}{Definition}
\newtheorem*{theorem}{Theorem}
\newtheorem*{lemma}{Lemma}
\newtheorem*{question}{Question}
\newtheorem*{preposition}{Preposition}
\newtheorem*{exercise}{Exercise}
\newtheorem*{challengeproblem}{Challenge Problem}
\newtheorem*{solution}{Solution}
\newtheorem*{remark}{Remark}
\usepackage{verbatimbox}
\usepackage{listings}
\title{Complex Analysis I: \\
Problem Set V}


\author{
Youngduck Choi \\
CILVR Lab \\
New York University\\
\texttt{yc1104@nyu.edu} \\
}


% The \author macro works with any number of authors. There are two commands
% used to separate the names and addresses of multiple authors: \And and \AND.
%
% Using \And between authors leaves it to \LaTeX{} to determine where to break
% the lines. Using \AND forces a linebreak at that point. So, if \LaTeX{}
% puts 3 of 4 authors names on the first line, and the last on the second
% line, try using \AND instead of \And before the third author name.

\newcommand{\fix}{\marginpar{FIX}}
\newcommand{\new}{\marginpar{NEW}}

\nipsfinalcopy % Uncomment for camera-ready version

\begin{document}


\maketitle

\begin{abstract}
This work contains the solutions to the problem set V
of Complex Analysis I 2015 at Courant Institute of Mathematical Sciences.
\end{abstract}

\bigskip

\begin{question}[1. 177.2]
\end{question}
\begin{solution}
Let $f$ be continuous on a closed bounded region $R$, and let it be
analytic and not constant throughout the interior of $R$. Assume 
that $f(z) \neq 0$ for $z \in R$. Let $g$ be a function on $R$,
defined by $g(z) = \dfrac{1}{f(z)}$ for $z \in R$. From the $g(z) = 
\dfrac{1}{f(z)}$ relation, we can deduce that
$g$ is also continuous, analytic and not constant throughout the interior
of $R$. Then, by the given corollary of the maximum modulus principle,
we have that the maximum value of $|g(z)|$ in $R$, which is always 
reached, occurs somewhere on the boundary of $R$ and never in the interior.
Observe that $|g(z)| = |\dfrac{1}{f(z)}| = \dfrac{1}{|f(z)|}$. Since
a modulus is strictly positive in this case, we have that maximum value
of $|g(z)|$ corresponds to the minimum value of $|f(z)|$. 
In other words,
the $z^*$, which is $\text{argmax} |g(z)|$ and lies on the boundary, is also
the $\text{argmin} |f(z)|$. Consequently, we have shown that a minimum
value is reached, and it occurs in the boundary of $R$ and never in
the interior. $\qed$ 
\end{solution}

\bigskip

\begin{question}[2. 177.4]
\end{question}
\begin{solution}
From the given hint, we have that
\eQb
|f(z)|^2 &=& sin^2(x) + sinh^2(y). 
\eQe
Observe that it reaches maximum with respect to $x$ on $\dfrac{\pi}{2}$
and with respect to $y$ on $1$, simply from the known properties of $sin$
and $sinh$ functions. Also, $(\dfrac{\pi}{2},1)$ is a feasible point.
Hence, we obtain that $|f(x)|^2$ reaches its maximum at $\dfrac{\pi}{2}
+ i$ on the boundary. $\qed$ 
\end{solution}

\bigskip

\begin{question}[3. 177.5]
\end{question}
\begin{solution}
Let $f(z) = u(x,y) + iv(x,y)$ be a function that is continuous on a closed
bounded region $R$ and not constant throughout the interior of $R$.
Consider an exponential function $g(x) = \exp(f(z))$. Observe that
$g$ is also continuous on a closed bounded region $R$ and non constant through
out the interior of $R$. Then, by the problem 2, we have $|g(x)|$,
which equals to $|\exp(u(x,y))|$, has a minimum value in $R$, which
occurs on the boundary of $R$, but never in the interior. Since
$\exp$ is an increasing function in reals, we have that the minima of 
$|\exp(u(x,y))|$ coincides with the minima of $u(x,y)$. Therefore, we have 
shown that the component function $u(x,y)$ has a minimum value in $R$,
which occurs on the boundary of $R$ and never in the interior. $\qed$ 
\end{solution}

\bigskip

\begin{question}[4. 195.3]
\end{question}
\begin{solution}
We wish to find the Maclaurin series expansion of the function
\eQb
f(z) &=& \dfrac{z}{z^4 + 4} = \dfrac{z}{4} \cdot 
\dfrac{1}{1 + (z^4/4)}.
\eQe
From the geometric series, we have
\eQb
\dfrac{1}{1-z} &=& \sum_{k=0}^{\infty} z^k, 
\eQe
for $|z| < 1$. Hence, by a change of variable,
we have
\eQb
\dfrac{1}{1+(\frac{z^4}{4})} &=& \sum_{k=0}^{\infty} (-\dfrac{z^4}{4})^k \\
&=& \sum_{k=0}^{\infty} \dfrac{(-1)^k}{4^k}z^{4k}, 
\eQe
for $|z| < \sqrt{2}$. 
It follows that
\eQb
f(z) &=& \dfrac{z}{4} 
\sum_{k=0}^{\infty} \dfrac{(-1)^k}{4^k}z^{4k} \\
&=& \sum_{k=0}^{\infty} \dfrac{(-1)^k}{4^{k+1}}z^{4k+1} \\
&=& \sum_{k=0}^{\infty} \dfrac{(-1)^k}{2^{2k+2}}z^{4k+1},
\eQe
for $|z| < \sqrt{2}$ as desired. $\qed$
\end{solution}

\bigskip

\begin{question}[5. 195.6]
\end{question}
\begin{solution}
Observe that we can write $tanh$ as 
\eQb
tanh &=& \dfrac{sinh}{cosh}.
\eQe
Observe that singularity happens at $cosh = 0$, which entails 
$z = (\dfrac{\pi}{2} + n\pi)i$. Therefore, we have analyticity 
for $|z| < \dfrac{\pi}{2}$, which is the largest circle within
which the Maclaurin series is defined. Taking derivatives of $tanh$ yields 
\eQb
tanh'(z) &=&  \dfrac{1}{cosh^2(x)} \\
tanh''(z) &=& -2\dfrac{sinh(x)}{cosh^3(x)}\\
tanh'''(z) &=& -2\dfrac{(1-2sinh(x))}{cosh^4(x)}. \\ 
\eQe
Substituting $0$ into $x$, we get 
\eQb
tanh(z) &=& z - \dfrac{1}{3}z^3 + ... 
\eQe
as desired for the first two nonzero terms of the series. $\qed$ 
\end{solution}

\pagebreak

\begin{question}[6. 195.11]
\end{question}
\begin{solution}
Observe that
\eQb
\dfrac{1}{4z -z^2} &=& \dfrac{1}{4z} \cdot \dfrac{1}{1-\frac{z}{4}}.
\eQe
By the geometric series, we have
\eQb
\dfrac{1}{1-\frac{z}{4}} &=& \sum_{k=0}^{\infty}\dfrac{z^k}{4^k},
\eQe
for $|z| < 4$. 
It follows that
\eQb
\dfrac{1}{4z -z^2} &=& \dfrac{1}{4z}
\sum_{k=0}^{\infty}\dfrac{z^k}{4^k} \\
&=& 
\dfrac{1}{4z} + \sum_{k=0}^{\infty}\dfrac{z^k}{4^{k+2}}, \\
\eQe
for $|z| < 4$ as desired. $\qed$
\end{solution}

\bigskip

\begin{question}[7. 205.5]
\end{question}
\begin{solution}
For $z \in D_1$, we have $|z| <1$. Hence, by the geometric series,
we obtain 
\eQb
\dfrac{1}{z-1} &=& - \dfrac{1}{1-z} = \sum_{k=0}^{\infty} z^n \\
-\dfrac{1}{z-2} &=& \dfrac{1}{2}\dfrac{1}{1- \frac{z}{2}} = 
\sum_{k=0}^{\infty} \dfrac{z^k}{2^{k+1}}. \\ 
\eQe
It follows that
\eQb
f(z) &=& -\sum_{k=0}^{\infty} z^k + \sum_{k=0}^{\infty} \dfrac{z^k}{2^{k+1}} \\
&=& \sum_{k=0}^{\infty} (-1 + 2^{-k-1}) z^k,
\eQe
for $z \in D_1$. 
\end{solution}

\bigskip

\begin{question}[8. 205.6]
\end{question}
\begin{solution}
By partial fraction decomposition, we have
\eQb
\dfrac{z}{(z-1)(z-3)} &=& 
\dfrac{3}{2}\cdot \dfrac{1}{z-3} - \dfrac{1}{2}
\cdot \dfrac{1}{z-1}.
\eQe
The above equality can be written as
\eQb
\dfrac{z}{(z-1)(z-3)} &=& -\dfrac{3}{4}\cdot 
\dfrac{1}{1- \frac{z-1}{2}} + \dfrac{1}{2}\cdot \dfrac{1}{1-z}.
\eQe
Since $0 < |z-1| < 2$, we have $0 < | \dfrac{z-1}{2} | 
< 1$. Therefore, by the geometric series, we have
\eQb
\dfrac{z}{(z-1)(z-3)} &=& -\dfrac{3}{4}
\sum_{k=0}^{\infty} \dfrac{(z-1)^k}{2^k} - \dfrac{1}{2(z-1)} \\
&=& -3\sum_{k=0}^{\infty} \dfrac{(z-1)^k}{2^{k+2}} - \dfrac{1}{2(z-1)}, \\ 
\eQe 
for $0 < |z-1| < 2$ as desired. $\qed$ 
\end{solution}

\bigskip

\begin{question}[9. 205.9]
\end{question}
\begin{solution}

\end{solution}

\bigskip

\begin{question}[10. 224.1]
\end{question}
\begin{solution}
Observe that
\eQb
e^z &=& \sum_{k=0}^{\infty} \dfrac{1}{k!}z^k \\
\dfrac{1}{z(z^2+1)} &=& 
\dfrac{1}{z}\sum_{k=0}^{\infty} 
(-z^2)^{k}, \\
&=& 
\sum_{k=0}^{\infty} 
(-1)^{k} z^{2k+1} \\
\eQe
$|z| < 1$. By multiplying out the first few terms manually, we obtain
\eQb
\dfrac{e^z}{z^2+1} &=& 
\dfrac{1}{z} + 1 - \dfrac{1}{2}z - \dfrac{5}{6}z^2 + ... ,
\eQe
for $|z| < 1$. $\qed$
\end{solution}

\bigskip

\begin{question}[11. 224.3]
\end{question}
\begin{solution}
We have previously shown that
\eQb
\sin(z) &=& \sum_{k=0}^{\infty} (-1)^k\dfrac{z^{2k+1}}{(2k+1)!} \\
 &=& \sum_{k=0}^{\infty} z - \dfrac{z^3}{3!} + \dfrac{z^5}{5!} + ..., \\ 
\eQe
for $|z| < \pi$. By doing the division of the first several terms by hand,
we obtain
\eQb
\csc(z) = \dfrac{1}{z} + \dfrac{1}{3!}z + [\dfrac{1}{(3!)^2} -
\dfrac{1}{5!}]z^3 + ... ,
\eQe
for $|z| < \pi$. $\qed$

\end{solution}

\bigskip

\begin{question}[12. 224.5]
\end{question}
\begin{solution}
From the Laurent series theorem, we have that
\eQb
b_1 &=& \dfrac{1}{2\pi i} \int_{C} \dfrac{1}{z^2 \sinh z} dz,
\eQe
where $b_1$ is the coefficient of the Laurent series for the $\dfrac{1}{z}$
and $C$ is the positively oriented unit circle $|z| = 1$.
Hence, by the given Laurent series, we obtain
\eQb
-\dfrac{1}{6} &=& \dfrac{1}{2\pi i} \int_{C} \dfrac{1}{z^2 \sinh z} dz.
\eQe
Re-arranging the terms yields
\eQb
\int_{C} \dfrac{1}{z^2 \sinh z} dz &=& -\dfrac{\pi}{3},
\eQe
where $C$ is the positively oriented unit circle $|z| = 1$. $\qed$

\end{solution}
\bigskip

\begin{question}[13. 224.8]
\end{question}
\begin{solution}
\end{solution}
\bigskip

\begin{question}[14. 224.9]
\end{question}
\begin{solution}
We have $\dfrac{1}{\cosh z}$. This function is singular when
$\cosh z = 0$, which occurs at $z = (\dfrac{\pi}{2} + n\pi)i$ 
for all $n \in \mathbb{Z}$. Therefore, the given function is 
analytic for the disk $|z| < \dfrac{\pi}{2}$. We have that
\eQb
\cosh(z) &=& \sum_{k=0}^{\infty} \dfrac{z^{2k}}{(2k!}. 
\eQe 
By doing the division by hand for the first few terms, we have
\eQb
\dfrac{1}{\cosh(z)} &=& 1 - \dfrac{1}{2!}z^2 + \dfrac{5}{4!}z^4
- \dfrac{61}{6!}z^6 + ....
\eQe
Therefore, we have shown that
\eQb
E_0 &=& 1 \\
E_2 &=& -1 \\
E_4 &=& 5 \\
E_6 &=& -61, \\
\eQe
as desired. $\qed$

\end{solution}
\bigskip

\end{document}
