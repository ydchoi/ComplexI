\documentclass{article} % For LaTeX2e
\usepackage{nips14submit_e,times}
\usepackage{amsmath}
\usepackage{amsthm}
\usepackage{amssymb}
\usepackage{mathtools}
\usepackage{hyperref}
\usepackage{url}
\usepackage{algorithm}
\usepackage[noend]{algpseudocode}
%\documentstyle[nips14submit_09,times,art10]{article} % For LaTeX 2.09

\usepackage{graphicx}
\usepackage{caption}
\usepackage{subcaption}

\def\eQb#1\eQe{\begin{eqnarray*}#1\end{eqnarray*}}
\def\eQnb#1\eQne{\begin{eqnarray}#1\end{eqnarray}}
\providecommand{\e}[1]{\ensuremath{\times 10^{#1}}}
\providecommand{\pb}[0]{\pagebreak}

\newcommand{\E}{\mathrm{E}}
\newcommand{\Var}{\mathrm{Var}}
\newcommand{\Cov}{\mathrm{Cov}}

\def\Qb#1\Qe{\begin{question}#1\end{question}}
\def\Sb#1\Se{\begin{solution}#1\end{solution}}

\newenvironment{claim}[1]{\par\noindent\underline{Claim:}\space#1}{}
\newtheoremstyle{quest}{\topsep}{\topsep}{}{}{\bfseries}{}{ }{\thmname{#1}\thmnote{ #3}.}
\theoremstyle{quest}
\newtheorem*{definition}{Definition}
\newtheorem*{theorem}{Theorem}
\newtheorem*{lemma}{Lemma}
\newtheorem*{question}{Question}
\newtheorem*{preposition}{Preposition}
\newtheorem*{exercise}{Exercise}
\newtheorem*{challengeproblem}{Challenge Problem}
\newtheorem*{solution}{Solution}
\newtheorem*{remark}{Remark}
\usepackage{verbatimbox}
\usepackage{listings}
\title{Complex Analysis I: \\
Problem Set VI}


\author{
Youngduck Choi \\
CILVR Lab \\
New York University\\
\texttt{yc1104@nyu.edu} \\
}


% The \author macro works with any number of authors. There are two commands
% used to separate the names and addresses of multiple authors: \And and \AND.
%
% Using \And between authors leaves it to \LaTeX{} to determine where to break
% the lines. Using \AND forces a linebreak at that point. So, if \LaTeX{}
% puts 3 of 4 authors names on the first line, and the last on the second
% line, try using \AND instead of \And before the third author name.

\newcommand{\fix}{\marginpar{FIX}}
\newcommand{\new}{\marginpar{NEW}}

\nipsfinalcopy % Uncomment for camera-ready version

\begin{document}


\maketitle

\begin{abstract}
This work contains the solutions to the problem set VI
of Complex Analysis I 2015 at Courant Institute of Mathematical Sciences.
\end{abstract}

\bigskip

\begin{question}[1. 237-2]
\end{question}
\begin{solution}
\textbf{(a)}
We have
\eQb
\dfrac{1}{z+z^2} &=& \dfrac{1}{z}\dfrac{1}{1+z} \\
&=& \dfrac{1}{z} (1 - z +z^2 \dots ) \\
&=& \dfrac{1}{z} - 1 + z \dots
\eQe
for $0 < |z| < 1$.
The coefficient of $\dfrac{1}{z}$ term is $1$. Hence, the residue is
$1$. \\

\smallskip

\textbf{(b)} 
We have
\eQb
z\cos(\dfrac{1}{z}) &=& z(1 - \dfrac{1}{2!}\dfrac{1}{z^2} + 
\dfrac{1}{4!}\dfrac{1}{z^4} \dots) \\
&=& z - \dfrac{1}{2!}\dfrac{1}{z} + \dfrac{1}{4!}\dfrac{1}{z^3} \dots \\
\eQe
for $|z| < \infty$. 
The coefficient of $\dfrac{1}{z}$ term is $0$. Hence, the residue is $0$.

\smallskip

\textbf{(c)} 
We have 
\eQb
\dfrac{z - \sin(z)}{z} &=& \dfrac{1}{z}\cdot\dfrac{1}{\sin(z)} \\
&=& \dfrac{1}{z}(z - (z - \dfrac{z^3}{3!} + \dfrac{z^5}{5!} \dots)) \\
\eQe
for $0 < |z| < \infty$. 
The coefficient of $\dfrac{1}{z}$ term is $0$. Hence, the residue is $0$.

\smallskip

\textbf{(d)} 
We have
\eQb
\dfrac{\cot(z)}{z^4} &=& \dfrac{1}{z^4}\cdot \dfrac{\cos(z)}{\sin(z)} \\
\eQe
By dividing the series representation of $\cos$ by $\sin$, we obtain
\eQb
\dfrac{\cos(z)}{\sin(z)} &=& \dfrac{1}{z} - \dfrac{z}{3} - \dfrac{z^3}{45}
\dots \\
\eQe
for $0 < |z| < \pi$. 
It follows that
\eQb
\dfrac{\cot(z)}{z^4} &=& \dfrac{1}{z^5} - \dfrac{1}{3}\cdot\dfrac{1}{z^3} - 
\dfrac{1}{45}\cdot\dfrac{1}{z} \dots  
\eQe
for $0 < |z| < \pi$. The coefficient of $\dfrac{1}{z}$ is $-\dfrac{1}{45}$.
Hence, the residue is $-\dfrac{1}{45}$.

\smallski[

\end{solution}


\bigskip

\begin{question}[1. Brown p.147-2]
\end{question}
\begin{solution}
\end{solution}

\bigskip

\begin{question}[1. Brown p.147-2]
\end{question}
\begin{solution}
\end{solution}

\bigskip

\begin{question}[1. Brown p.147-2]
\end{question}
\begin{solution}
\end{solution}

\bigskip

\begin{question}[1. Brown p.147-2]
\end{question}
\begin{solution}
\end{solution}

\end{document}
