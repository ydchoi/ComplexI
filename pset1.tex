\documentclass{article} % For LaTeX2e
\usepackage{nips14submit_e,times}
\usepackage{amsmath}
\usepackage{amsthm}
\usepackage{amssymb}
\usepackage{mathtools}
\usepackage{hyperref}
\usepackage{url}
\usepackage{algorithm}
\usepackage[noend]{algpseudocode}
%\documentstyle[nips14submit_09,times,art10]{article} % For LaTeX 2.09

\usepackage{graphicx}
\usepackage{caption}
\usepackage{subcaption}

\def\eQb#1\eQe{\begin{eqnarray*}#1\end{eqnarray*}}
\def\eQnb#1\eQne{\begin{eqnarray}#1\end{eqnarray}}
\providecommand{\e}[1]{\ensuremath{\times 10^{#1}}}
\providecommand{\pb}[0]{\pagebreak}

\newcommand{\E}{\mathrm{E}}
\newcommand{\Var}{\mathrm{Var}}
\newcommand{\Cov}{\mathrm{Cov}}

\def\Qb#1\Qe{\begin{question}#1\end{question}}
\def\Sb#1\Se{\begin{solution}#1\end{solution}}

\newenvironment{claim}[1]{\par\noindent\underline{Claim:}\space#1}{}
\newtheoremstyle{quest}{\topsep}{\topsep}{}{}{\bfseries}{}{ }{\thmname{#1}\thmnote{ #3}.}
\theoremstyle{quest}
\newtheorem*{definition}{Definition}
\newtheorem*{theorem}{Theorem}
\newtheorem*{lemma}{Lemma}
\newtheorem*{question}{Question}
\newtheorem*{preposition}{Preposition}
\newtheorem*{exercise}{Exercise}
\newtheorem*{challengeproblem}{Challenge Problem}
\newtheorem*{solution}{Solution}
\newtheorem*{remark}{Remark}
\usepackage{verbatimbox}
\usepackage{listings}
\title{Complex Analysis: \\
Problem Set I}


\author{
Youngduck Choi \\
CILVR Lab \\
New York University\\
\texttt{yc1104@nyu.edu} \\
}


% The \author macro works with any number of authors. There are two commands
% used to separate the names and addresses of multiple authors: \And and \AND.
%
% Using \And between authors leaves it to \LaTeX{} to determine where to break
% the lines. Using \AND forces a linebreak at that point. So, if \LaTeX{}
% puts 3 of 4 authors names on the first line, and the last on the second
% line, try using \AND instead of \And before the third author name.

\newcommand{\fix}{\marginpar{FIX}}
\newcommand{\new}{\marginpar{NEW}}

\nipsfinalcopy % Uncomment for camera-ready version

\begin{document}


\maketitle

\begin{abstract}
This work contains the solutions to the problem set I
of Complex Analysis I 2015 at Courant Institute of Mathematical Sciences.
\end{abstract}

\begin{question}[1. Brown p61. 1]
\end{question}
\begin{solution}
We wish to give a direct proof that 
\eQb
\dfrac{dw}{dz} = 2z \> \> \text{when } \> \> w = z^2,
\eQe
using the definition $(3)$ in section $19$, which is
\eQb
\dfrac{dw}{dz} = \underset{\triangle z \to 0 }{\lim} \dfrac{\triangle w }{\triangle z}.
\eQe
We proceed to compute $\underset{\triangle z \to 0}{\lim} \dfrac{\triangle w }{\triangle z}$,
given that $w = z^2$.
\eQb
\underset{\triangle z \to 0}{\lim} \dfrac{\triangle w}{\triangle z} &=& 
\underset{\triangle z \to 0}{\lim} (({\triangle z} + z)^2 - z^2) 
\dfrac{1}{\triangle z} \\
&=& \underset{\triangle z \to 0}{\lim} ({\triangle z}^2 + 2z{\triangle z})
\dfrac{1}{\triangle z} \\ 
&=& \underset{\triangle z \to 0}{\lim} {\triangle z} + 2z \\
&=& 2z.
\eQe
Therefore, we have shown that $\dfrac{dw}{dz} = 2z$. $\qed$
\end{solution}

\bigskip

\begin{question}[2. Brown p61. 2]
\end{question}
\begin{solution} We differentiate four given functions of $z$. \\

\smallskip

\textbf{(a)} We wish to differentiate $f(z) = 3z^2 - 2z + 4$. Simply applying the power rule,
we obtain $f^{'}(z) = 6z - 2$. \\

\smallskip

\textbf{(b)} We wish to differentiate $f(z) = (2z^2 + i)^5$. Applying the chain rule,
we obtain 
\eQb
f^{'}(z) &=& 5(2z^2 + i)^4 \cdot 4z \\
&=& 20z(2z^2 + i)^4.
\eQe

\smallskip

\textbf{(c)} We wish to differentiate $f(z) = \dfrac{z-1}{2z+1} \> (z \neq \dfrac{1}{2})$. 
Applying the quotient rule, we obtain 
\eQb
f^{'}(z) &=& \dfrac{(2z+1)(1) - (z-1)(2)}{(2z+1)^2} \\ 
&=& \dfrac{3}{(2z+1)^2},
\eQe
for $z \neq \dfrac{1}{2}$.

\smallskip

\textbf{(d)} We wish to differentiate $f(z) = \dfrac{(1+z^2)^4}{z^2} \> (z \neq 0)$. Applying
the quotient rule, we obtain
\eQb
f^{'}(z) &=& \dfrac{(z^2)(\dfrac{d}{dz} (1+z^2)^4) - (1+z^2)^4 (2z)}{z^4},
\eQe
for $z \neq 0$.
Using the chain rule to resolve $\dfrac{d}{dz} (1+z^2)^4$ term, we finally get
\eQb
f^{'}(z) &=& \dfrac{(z^2)(4)(1+z^2)^3(2z) - (1+z^2)^4(2z)}{z^4} \\
&=& \dfrac{8z^3(1+z^2)^3 - 2z(1+z^2)^4}{z^4} \\
&=& \dfrac{8z^2(1+z^2)^3 - 2(1+z^2)^4}{z^3} \\
&=& \dfrac{2(1+z^2)^3(3z^2 - 1)}{z^3}, \\
\eQe
for $z \neq 0$.
\end{solution}

\bigskip

\begin{question}[3. Brown p76. 4]
\end{question}
\begin{solution}
We determine the singular points of the three given functions of $z$.

\smallskip

\textbf{(a)} We wish to determine the singular points of $f(z) = \dfrac{2z + 1}{z(z^2+1)}$.
The singular points are $z = 0, \pm i$.
\smallskip

\textbf{(b)} We wish to determine the singular points of $f(z) = \dfrac{z^3 + i}{z^2 - 3z + 2}$.
Notice that the function definition can be factorized as
\[
f(z) = \dfrac{z^3 + i}{(z-2)(z-1)}.
\]
The singular points are $z = 1,2$.
\smallskip

\textbf{(c)} We wish to determine the singular points of $f(z) = \dfrac{z^2 + 1}{(z+2)
(z^2 + 2z + 2)}$. The singular points are $z=-2, 1 \pm i$. 


\end{solution}

\pagebreak

\begin{question}[4. Brown p90. 5]
\end{question}
\begin{solution}
The terms $| \mathrm{exp}(2z + i) |$ and $| \mathrm{exp}(iz^2 )|$ can be written as
\eQnb \label{eq:xy}
|\mathrm{exp}(2z + i) | = |\mathrm{exp}(2x + i(2y+1) ) | = e^{2x} \\
|\mathrm{exp}(iz^2) = |\mathrm{exp}(-2xy + i(x^2 - y^2))| = e^{2xy}.
\eQne

By the triangle-inequality, we have that 
\eQb
|\mathrm{exp}(2z + i) + \mathrm{exp}(iz^2 )| \leq 
|\mathrm{exp}(2z + i) | + |\mathrm{exp} (iz^2 ) |. 
\eQe

Using the \ref{eq:xy} substitution, we can conclude that
\eQb
|\mathrm{exp}(2z + i) + \mathrm{exp}(iz^2) | \leq \mathrm{e}^{2x} + \mathrm{e}^{-2xy}.
\eQe
$\qed$

\end{solution}

\bigskip

\begin{question}[5. Brown p185. 4]
\end{question}
\begin{solution}
We have the following summation formulation:
\eQnb \label{eq:sum} 
\sum_{n=1}^{\infty} z^n &=& \dfrac{z}{1-z},
\eQne
whenever $|z| < 1$. Substituting $r\mathrm{e}^{i\theta}$ for $z$
and separating the real and imaginary parts, we can re-write the LHS as
\begin{eqnarray} \label{eq:LHS}
\sum_{n=1}^{\infty} z^n &=& \sum_{n=0}^{\infty} (r\mathrm{e}^{i\theta})^n \nonumber \\
&=& \sum_{n=1}^{\infty} r^n \mathrm{e}^{in \theta} \nonumber \\
&=& \sum_{n=1}^{\infty} r^n \mathrm{cos}(n \theta ) + i \sum_{n=0}^{\infty} r^n \mathrm{sin}(n \theta). 
\end{eqnarray} 
Now, multiplying both denominator and numerator by the conjugate of $1 -z$, $1 - \overline{z}$,
we can re-write the RHS as
\eQb
\dfrac{z}{1-z} &=& \dfrac{z(1-\overline{z})}{(1-z)(1-\overline{z})} \\
&=& \dfrac{z - z\overline{z}}{1 + z\overline{z} - (z + \overline{z})}. \\
\eQe
Substituting $z =r\mathrm{cos}(\theta ) + i\mathrm{sin}(\theta )$,
$z\overline{z} = r^2$, and  $z + \overline{z} = 2r\mathrm{cos}(\theta )$ to the last expression,
and separating the real and imaginary parts,
we obtain
\begin{eqnarray} \label{eq:RHS}
\dfrac{z}{1-z} &=& \dfrac{r\mathrm{cos}(\theta ) - r^2 + ir\mathrm{sin}(\theta ) }
{1 + r^2 - 2r\mathrm{cos}(\theta )} \nonumber \\
&=& \dfrac{r\mathrm{cos}(\theta ) - r^2}{1+r^2 - 2r\mathrm{cos}(\theta )} 
+ \dfrac{r\mathrm{sin}(\theta )}{1 + r^2 -2r\mathrm{cos}(\theta ) }i.  
\end{eqnarray}
Substituting \ref{eq:RHS} and \ref{eq:LHS} to \ref{eq:sum}, we obtain
\eQnb
\sum_{n=1}^{\infty} r^n \mathrm{cos}(n \theta ) + i \sum_{n=0}^{\infty} r^n \mathrm{sin}(n \theta) 
&=& \dfrac{r\mathrm{cos}(\theta ) - r^2}{1+r^2 - 2r\mathrm{cos}(\theta )} 
+ \dfrac{r\mathrm{sin}(\theta )}{1 + r^2 -2r\mathrm{cos}(\theta ) }i.  
\eQne
By the Theorem from section $61$, we know that the real and imaginary part of the series
must equal the real and imaginary part of the convergent value respectively. Hence, we have that
\eQb
\sum_{n=1}^{\infty} r^n \mathrm{cos}(n \theta) = \dfrac{r\mathrm{cos}(\theta ) - r^2}{1 - 
2r\mathrm{cos}(\theta ) + r^2} \\
\sum_{n=1}^{\infty} r^n \mathrm{sin} (n \theta) = \dfrac{r\mathrm{sin}(\theta )}{1 - 2r\mathrm{cos}
(\theta ) + r^2},
\eQe
when $0 < r < 1$. $\qed$

\end{solution}

\pagebreak

\begin{question}[6. Radius of Convergence I]
\end{question}
\begin{solution}
We wish to find the radius of convergence of the power series 
\[
\sum_{k=1}^{\infty} \dfrac{(-1)^k}{k} z^{k(k+1)}.
\]
Let $s_k$ denote the $k$th term of the above power series.
From the ratio test, we know that a series $\sum_{k=1} s_k$ converges if 
$\underset{k \to \infty}{\limsup} | \dfrac{s_{k+1}}{s_k} | < 1$ and diverges
if $|\dfrac{s_{k+1}}{s_k} | \geq 1$ for all $k \geq k_0$, where $n_0$ is some
fixed integer. We proceed to compute $\underset{k \to \infty}{\limsup} | \dfrac{s_{k+1}}{s_k}|$.

\eQb
\limsup_{k \to \infty} | \dfrac{s_{k+1}}{s_k} | &=& \limsup_{k \to \infty}
|\dfrac{\dfrac{(-1)^{k+1}}{k+1}z^{(k+1)(k+2)}}{\dfrac{(-1)^k}{k}z^{k(k+1)}}| \\
&=& \limsup_{k \to \infty} |\dfrac{-k}{k+1}z^2| \\
&=& \limsup_{k \to \infty} | \dfrac{k}{k+1} | |z^2 | \\
&=& |z^2|\limsup_{k \to \infty} |\dfrac{k}{k+1}| \\
&=& |z|^2.
\eQe

For $|z| < 1$, we have that $\underset{k \to \infty}{\limsup} | \dfrac{s_{k+1}}{s_k} | = |z|^2 < 1$.
Hence, the series converges when $|z| < 1$. For $|z| > 1$, we have that 
$\underset{k \to \infty}{\limsup} | \dfrac{s_{k+1}}{s_k} | = |z|^2 > 1$. Hence, the series diverges
when $|z| > 1$. Therefore, we have that the radius of convergence of the given sequence is $1$,
with the center being the origin.\\

\smallskip

Now, we discuss the convergence for $z = 1$ and $-1$, and $i$. For $z = 1,-1$, 
as $z^{k(k+1)} = 1$ for all $k$, the series simplifies to
\[
\sum_{k=1}^{\infty} \dfrac{(-1)^k}{k}.
\]

Notice that $-\dfrac{1}{k} +\dfrac{1}{k+1} = -\dfrac{1}{k(k+1)}$ for $k \geq 1$. Hence, the above series
can be re-written in the following way: 
\[
\sum_{k=1}^{\infty} \dfrac{(-1)^k}{k} = -\sum_{k=1}^{\infty}\dfrac{1}{(2k-1)(2k)}.
\]

Observe that
\[
-\dfrac{1}{(2k-1)(2k-1)}\leq -\dfrac{1}{(2k-1)(2k)} \leq 0, 
\]
for $k \leq 1$.
As $\sum_{k=1}^{\infty}\dfrac{1}{k^2}$ converges to $0$,
$\sum_{k=0}^{\infty} -\dfrac{1}{(2k-1)(2k-1)}$ converges to $0$, and by the squeeze theorem,
we have that $\sum_{k=1}\dfrac{1}{(2k-1)(2k)}$ converges to $0$. Thus, the given series is
convergent for $z = 1$ and $-1$. \\

\pagebreak

For $z = i$, 
\end{solution}

\bigskip

\begin{question}[7. Radius of Convergence II]
\end{question}
\begin{solution}
We wish to find the radius of convergence of the power series 
\eQb
\sum_{k=0}^{\infty} \dfrac{k^2}{4^k + 5k} z^k.
\eQe

Let $s_k$ denote the $k$th term of the above power series. From the root test,
we know that a series $\sum_{k=1}^{\infty}$ converges if $\underset{k \to \infty}{\limsup}
|s_k|^{\frac{1}{k}} < 1$ and diverges if $\underset{k \to \infty}{\limsup} |s_k|^{\frac{1}{k}}
> 1$. We proceed to compute $\underset{k \to \infty}{\limsup} | s_k |^{\frac{1}{k}}$.
\eQb
\underset{k \to \infty}{\limsup} | s_k |^{\frac{1}{k}} &=& 
\underset{k \to \infty}{\limsup} |\dfrac{k^2}{4^k + 5k}z^k|^{\frac{1}{k}} \\
&=& \underset{k \to \infty}{\limsup} 
|\dfrac{k^2}{4^k + 5k}|^{\frac{1}{k}}|z^k|^{\frac{1}{k}} \\
&=& |z| \underset{k \to \infty}{\limsup} |(\dfrac{k^2}{4^k + 5k})^{\frac{1}{k}}| \\
&=& \dfrac{1}{4} |z|.
\eQe

For $|z| < 4$, we have that $\underset{k \to \infty}{\limsup} |s_k|^{\frac{1}{k}}
= \dfrac{1}{4}|z| < 1$. Hence, the series converges when $|z| < 4$.
For $|z| > 4$, we have that $\underset{k \to \infty}{\limsup} |s_k|^{\frac{1}{k}}
= \dfrac{1}{4}|z| > 1$. Hence, the series diverges when $|z| > 4$. Therefore,
we have that the radius of convergence of the given sequence is $4$, with the
center being the origin. $\qed$

\end{solution}


\end{document}
