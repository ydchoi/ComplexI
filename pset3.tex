\documentclass{article} % For LaTeX2e
\usepackage{nips14submit_e,times}
\usepackage{amsmath}
\usepackage{amsthm}
\usepackage{amssymb}
\usepackage{mathtools}
\usepackage{hyperref}
\usepackage{url}
\usepackage{algorithm}
\usepackage[noend]{algpseudocode}
%\documentstyle[nips14submit_09,times,art10]{article} % For LaTeX 2.09

\usepackage{graphicx}
\usepackage{caption}
\usepackage{subcaption}

\def\eQb#1\eQe{\begin{eqnarray*}#1\end{eqnarray*}}
\def\eQnb#1\eQne{\begin{eqnarray}#1\end{eqnarray}}
\providecommand{\e}[1]{\ensuremath{\times 10^{#1}}}
\providecommand{\pb}[0]{\pagebreak}

\newcommand{\E}{\mathrm{E}}
\newcommand{\Var}{\mathrm{Var}}
\newcommand{\Cov}{\mathrm{Cov}}

\def\Qb#1\Qe{\begin{question}#1\end{question}}
\def\Sb#1\Se{\begin{solution}#1\end{solution}}

\newenvironment{claim}[1]{\par\noindent\underline{Claim:}\space#1}{}
\newtheoremstyle{quest}{\topsep}{\topsep}{}{}{\bfseries}{}{ }{\thmname{#1}\thmnote{ #3}.}
\theoremstyle{quest}
\newtheorem*{definition}{Definition}
\newtheorem*{theorem}{Theorem}
\newtheorem*{lemma}{Lemma}
\newtheorem*{question}{Question}
\newtheorem*{preposition}{Preposition}
\newtheorem*{exercise}{Exercise}
\newtheorem*{challengeproblem}{Challenge Problem}
\newtheorem*{solution}{Solution}
\newtheorem*{remark}{Remark}
\usepackage{verbatimbox}
\usepackage{listings}
\title{Complex Analysis I: \\
Problem Set III}


\author{
Youngduck Choi \\
CILVR Lab \\
New York University\\
\texttt{yc1104@nyu.edu} \\
}


% The \author macro works with any number of authors. There are two commands
% used to separate the names and addresses of multiple authors: \And and \AND.
%
% Using \And between authors leaves it to \LaTeX{} to determine where to break
% the lines. Using \AND forces a linebreak at that point. So, if \LaTeX{}
% puts 3 of 4 authors names on the first line, and the last on the second
% line, try using \AND instead of \And before the third author name.

\newcommand{\fix}{\marginpar{FIX}}
\newcommand{\new}{\marginpar{NEW}}

\nipsfinalcopy % Uncomment for camera-ready version

\begin{document}


\maketitle

\begin{abstract}
This work contains the solutions to the problem set III
of Complex Analysis I 2015 at Courant Institute of Mathematical Sciences.
\end{abstract}

\bigskip

\begin{question}[1. Brown p.95-4]
\end{question}
\begin{solution}
We are given the following branch:
\eQb
log(z) &=& ln(r) + i\theta \> \text{ and } \> r > 0 , \dfrac{3\pi}{4} < \theta < \dfrac{11\pi}{4}.
\eQe
With the given branch, the computations yield
\eQb
log(i^2) &=& log(-1) = ln(1) + i\pi \\
2log(i) &=& 2(ln(1) +i \dfrac{5\pi}{2}) = i\dfrac{5\pi}{2}.
\eQe
Hence, we have that $log(i^2) = 2log(i)$ for this particular branch. $\qed$
\end{solution}



\bigskip

\begin{question}[2. Brown p.95-11]
\end{question}
\begin{solution}
We wish to show that $ln(x^2 + y^2)$ is harmonic.
Firstly, we can compute the partials as follow: 
\eQb
u_x &=& \dfrac{2x}{x^2 + y^2} \\
u_{xx} &=& \dfrac{2y^2 - 2x^2}{(x^2 + y^2)^2}, \\
\eQe
for $x,y \neq 0$.
By symmtry, we also have
\eQb
u_y &=& \dfrac{2y}{x^2 + y^2} \\
u_{yy} &=& \dfrac{2x^2 - 2y^2}{(x^2 + y^2)^2}, \\
\eQe
for $x,y \neq 0$. Hence, we have that $u_{xx} + u_{yy} = 0$ for $x,y \neq 0$,
and consequently $ln(x^2+y^2)$ is harmonic. Now, we show that $ln(x^2+y^2)$ is harmonic
in a different way. 
\end{solution}

\bigskip

\begin{question}[3. Brown p.103-1]
\end{question}
\begin{solution}
We can re-write the expression $(1+i)^{i}$ as
\eQb
(1+i)^{i} &=& \exp{(i\ln(1+i))} \\ 
&=& \exp{(i(\ln(\sqrt{2}) + i(\dfrac{\pi}{4}+ 2\pi n)))} \\
&=& \exp{(-\dfrac{\pi}{4} - 2\pi n)} \exp{(i(\dfrac{\ln(2)}{2}))} \\
&=& \exp{(-\dfrac{\pi}{4} + 2\pi n)} \exp{(i(\dfrac{\ln(2)}{2}))}
\eQe
for $n = 0, \pm 1, \pm 2, ...$ . Now, we can also re-write the expression $\dfrac{1}{i^{2i}}$
as 
\eQb
\dfrac{1}{i^{2i}} &=& e^{-2i} \\
&=& e^{-2i\log(i)} \\
&=& \exp{(-2i(\ln(1) + i(\dfrac{\pi}{2} + 2n\pi))} \\
&=& \exp{((4n+1)\pi)},
\eQe 
for $n = 0, \pm 1, \pm 2, ...$ as desired. $\qed$
\end{solution}

\bigskip

\begin{question}[4. Brown p.133-3]
\end{question}
\begin{solution}
We are given a function $f(z) = \pi \exp{(\pi\bar{z})}$, contour $C$ as the boundary of
the square with vertices at the points $0, 1, 1 + i, i$, and the orientation of $C$
being the counter-clockwise direction. Separating the integral with 4 different legs, we have
\eQb
\int_{C} \pi \exp{(\pi \bar{z})} dz &=&
\int_{C_1} \pi \exp{(\pi \bar{z})} dz +
\int_{C_2} \pi \exp{(\pi \bar{z})} dz \\ 
&+&
\int_{C_3} \pi \exp{(\pi \bar{z})} dz +
\int_{C_4} \pi \exp{(\pi \bar{z})} dz, 
\eQe
where the legs can be written as 
\eQb
C_1 : z &=& x (0 \leq x \leq 1) \\
C_2 : z &=& 1+ iy (0 \leq y \leq 1) \\
C_3 : z &=& 1-x + i (0 \leq x \leq 1) \\ 
C_4 : z &=& i(1-y) (0 \leq y \leq 1).
\eQe
Simplifying the leg integrals with their particular values, we obtain
\eQb
\int_{C_1} \pi \exp{(\pi \bar{z})} dz &=&
\pi \int_{0}^{1} \exp{(\pi x)} dx \\
&=& e^{\pi} - 1 \\
\int_{C_2} \pi \exp{(\pi \bar{z})} dz &=&
\pi \int_{0}^{1} \exp{(\pi (1-iy))} dy \\
&=& 2e^{\pi} \\
\int_{C_3} \pi \exp{(\pi \bar{z})} dz &=&
\pi \int_{0}^{1} \exp{(\pi (1-x-i))} dx \\
&=& e^{\pi} - 1 \\
\int_{C_4} \pi \exp{(\pi \bar{z})} dz &=&
\pi \int_{0}^{1} \exp{(\pi i(y-1))} dy \\
&=& -2. 
\eQe
Hence, adding them up, we obtain
\eQb
\int_{C} \pi e^{\pi \bar{z}} dz = 4(e^\pi -1),
\eQe
as desired. $\qed$

\end{solution}

\bigskip

\begin{question}[5. Brown p.133-8]
\end{question}
\begin{solution}
\end{solution}

\bigskip

\begin{question}[6. Brown p.138-1]
\end{question}
\begin{solution}
\textbf{(a)} We wish to show that $|\int_{C} \dfrac{z+4}{z^3 - 1} dz | \leq \dfrac{6\pi}{7}$,
where $C$ is a quarter circle from $2$ to $2i$.
We have the length of the contour is $\pi$. 
Now, we compute the upper bound of $|f(z)|$ along the contour: 
\eQb
\left| \dfrac{z+4}{z^3 -1 } \right| &=& \dfrac{|z+4|}{|z^3 - 1|}.
\eQe
By the triangle inequality, we have
\eQb
|z+4| \leq |z| + |4| = 6.
\eQe
Again, by the triangle inequality, we have
\eQb
|z^3 - 1| \geq |z^3| - |1| = |z|^3 - 1 = 8 - 1 = 7.
\eQe
It follows that
\eQb 
\dfrac{1}{|z^3 - 1|}  \leq \dfrac{1}{7}.
\eQe
Consequently, combining the two inequalities yields 
\eQb
\left| \dfrac{z+4}{z^3 -1} \right| \leq \dfrac{6}{7}.
\eQe
Since $f$ is piece-wise continous on $C$, we have 
\eQb
\left| \int_{C} \dfrac{z+}{z^3 - 1} dz \right|  \leq \dfrac{6\pi}{7},
\eQe
as desired. 

\smallskip

\textbf{(b)} We wish to show that 
$\left| \int_{C} \dfrac{dz}{z^2 -1} \right| \leq \dfrac{\pi}{3}$, where
$C$ is a quarter circle from $2$ to $2i$. 
The length of the contour as $\pi$. We now compute the upper bound
of $|f(z)|$ along the contour, which can be written as $\left| \dfrac{1}{z^2-1} \right|$.
By the triangle inequality, we have
\eQb
|z^2 - 1| \geq |z^2| - |1| = 4 - 1 = 3.
\eQe
Hence, we have
\eQb
\left| \dfrac{1}{z^2 -1} \right| = \dfrac{1}{\left| z^2 -1 \right|} \leq \dfrac{1}{3}. 
\eQe 
Since $f$ is piece-wise continuous on $C$, we have
\eQb
\left| \int_{C} \dfrac{dz}{z^2 -1} \right| \leq \dfrac{\pi}{3}, 
\eQe 
as desired. $\qed$

\end{solution}

\bigskip

\begin{question}[7. Brown p.138-2]
\end{question}
\begin{solution}
We wish to show that $\left| \int_{C}  \dfrac{dz}{z^4} \right| \leq 4\sqrt{2}$, where
$C$ is a straight line from $i$ to $1$. The contour length is $\sqrt{2}$.
Observe that of all the points on the line segment, the midpoint is the closest to the origin,
that distance being $d = \dfrac{\sqrt{2}}{2}$. Hence, we obtain
\eQb
|z| \geq \dfrac{\sqrt{2}}{2}.
\eQe
Consequently, it follows that 
\eQb
|z|^4 \geq \dfrac{1}{4} \text { and } \dfrac{1}{|z|^4} \leq 4.
\eQe
Since $f$ is piece-wise continuous on $C$, we have 
\eQb
\left| \int_{C} \dfrac{dz}{z^4} \right| \leq 4\sqrt{2}, 
\eQe
as desired. $\qed$
\end{solution}

\bigskip

\begin{question}[8. Brown p.138-3]
\end{question}
\begin{solution}
We wish to show that $\left| \int_{C} (e^z - \bar{z}) dz \right| \leq 60$, where $C$ is the
boundary of the triangle with vertices at the points $0$, $3i$, and $-4$. Notice that
the contour length is simply $12$, as the triangle is a $3-4-5$ triangle. Now, we wish to 
compute an upper bound of $|f(z)|$ along the contour. Observe that
\eQb
|e^z - \bar{z}| \leq e^x + \sqrt{x^2 + y^2}, 
\eQe  
where $z = x + iy$. As $x \leq 0$, we have that $e^x \leq 1$. Furthermore, 
\end{solution}

\bigskip

\begin{question}[9. Brown p.138-5]
\end{question}
\begin{solution}
We wish to show the following inequality:
\eQb
\left| \int_{C_R} \dfrac{\log(z)}{z^2} dz \right| \leq
2\pi(\dfrac{\pi + \ln(R)}{R}),
\eQe
where $C_R$ denotes the contour along the circle $|z| = R (R>1)$. Hence, the length of the contour
is simply $2\pi R$. Now, we compute an upper bound of $\left| \dfrac{\log(z)}{z^2} \right|$ along
the contour. Basic algebraic manipulations and using triangle inequality, we have
\eQb
\left| \dfrac{\log(z)}{z^2} \right| &=& \dfrac{| \log(z) |}{|z^2|} = \dfrac{|\ln(R) + i\theta|}{R^2} \\
&\leq& \dfrac{\ln(R) + |\theta|}{R^2} \leq \dfrac{\ln(R) + \pi}{R^2}, 
\eQe 
as $-\pi \leq \theta \leq \pi$. Since the given function is piecewise continuous on $C$, we obtain 
\eQb
\left| \int_{C_R} \dfrac{\log(z)}{z^2} dz \right| \leq
2\pi(\dfrac{\pi + \ln(R)}{R}),
\eQe
as desired. By using the l'Hospital's rule. we have
\eQb
\underset{R \to \infty}{\lim} 2\pi(\dfrac{\pi + \ln(R)}{R}) &=& 
\underset{R \to \infty}{\lim} \dfrac{1}{R} = 0.
\eQe
Therefore, as $R \to \infty$ the upper bound of the absolute value of the integral tends to $0$.
Consequently, we see that the integral must tend to $0$ as well. $\qed$
\end{solution}

\bigskip

\begin{question}[10. Brown p.138-8]
\end{question}
\begin{solution}
\textbf{(a)} 
On the vertical side of the square, we have $ x = (N + \dfrac{1}{2})\pi$. Therefore, 
$\mathrm{sin}(x) = -1$ or $1$. Hence, as we have $|\mathrm{sin}(z)| \geq |\mathrm{sin}(x)|$,
we obtain $|\mathrm{sin}(z)| \geq 1$. On the horizontal side of the square, we have
$y = (N + \dfrac{1}{2})\pi$. Therefore, $\mathrm{sinh}(y) = \mathrm{sinh}(\pm \dfrac{1}{2}\pi)$.
Hence, as we have $|\mathrm{sin}(z)| \geq |\mathrm{sinh}(y)|$, we obtain 
$|\mathrm{sin}(z)| \geq |\mathrm{sinh}(\dfrac{\pi}{2})|$. Cosnequently, there is a positive constant $A$,
independent of $N$, such that $|\mathrm{sin}(z)| \leq A$ for all points $z$ lying on the contour
$C_N$.

\bigskip
 
\textbf{(b)} We wish to show that 
$\left| \int_{C_N} \dfrac{1}{z^2 sin(z)} dz \right| 
\leq \dfrac{16}{(2N+1)\pi A}$.
 The length of the $C_N$ contour is $8(N+\frac{1}{2})\pi$. Now,
we compute an upper bound of $\left| \dfrac{1}{z^2 sin(z)} \right|$ along the contour.
We have that $|z^2| \geq ((N+\frac{1}{2})\pi)^2$ and $|sin(z)| \geq A$ on $C_N$. 
It follows that
\eQb
\left| \dfrac{1}{z^2 sin(z)} \right| &=& \dfrac{1}{|z^2||sin(z)|} \\
&\leq& \dfrac{1}{((N+\frac{1}{2})\pi)^2 A},
\eQe
holds on $C_N$. Therfore, as $\dfrac{1}{z^2 sin(z)}$ is piece-wise continuous, we have
\eQb
\left| \int_{C_N} \dfrac{1}{z^2 sin(z)} dz \right| &\leq& \dfrac{8(N+\frac{1}{2})\pi}
{((N+\frac{1}{2})\pi)^2 A} \\ 
&=& \dfrac{16}{(2N+1)\pi A},
\eQe
as desired. $\qed$

\end{solution}


\end{document}



